\begin{thebibliography}{99}
    %
    \bibitem{DALIE}
        LBL diffential algebra package and LieLib routines courtesy of \'{E}.~Forest.
    %
    \bibitem{Ripken95}
        G.~Ripken and F.~Schmidt,
        ``A symplectic six-dimensional thin-lens formalism for tracking'',
        CERN SL 95--12 (AP)(1995), DESY 95--063 (1995);\\
        G.~Ripken and F.~Schmidt,
        ``Construction of Nonlinear Symplectic Six-Dimensional Thin-Lens Maps by Exponentiation'',
        DESY 95--189 (1995), \url{http://cern.ch/Frank.Schmidt/report/ripken2.pdf};\\
        D.P.~Barber, K.~Heinemann, G.~Ripken and F.~Schmidt,
        ``Symplectic Thin\,-\,Lens Transfer Maps for SixTrack: Treatment of  Bending Magnets in Terms of the Exact Hamiltonian'',
        DESY 96--156 (1995), \url{http://cern.ch/Frank.Schmidt/report/ripken3.pdf}.
    %
    \bibitem{RACETRACK}
        A.~Wrulich,
        ``RACETRACK, A computer code for the simulation of nonlinear motion in accelerators'',
        DESY 84--026 (1984).
    %
    \bibitem{FASTRAC}
        B.~Leemann and \'{E}.~Forest,
        ``Brief description of the tracking codes FASTRAC and THINTRAC'',
        SSC Note SSC--133.
    %
    \bibitem{Ripken85}
        G.~Ripken,
        ``Nonlinear canonical equations of coupled synchro-betatron motion and their solution within the framework of a nonlinear 6-dimensional (symplectic) tracking program for ultra-relativistic protons'',
        DESY 85--084 (1985).
    %
    \bibitem{Ripken87}
        D.P.~Barber, G.~Ripken and F.~Schmidt,
        ``A nonlinear canonical formalism for the coupled synchro-betatron motion of protons with arbitrary energy'',
        DESY 87--036 (1987);\\
        G.~Ripken and F.~Schmidt,
        ``A symplectic six-dimensional thin-lens formalism for tracking'',
        CERN/SL/95--12 (AP), DESY 95--063 (1995), \url{http://cern.ch/Frank.Schmidt/report/ripken.pdf};
        %K.~Heinemann,
    %
    \bibitem{HBOOK}
        R.~Brun and D.~Lienart,
        ``HBOOK User Guide'',
        CERN Program Library Y250 (1987).
    %
    \bibitem{HPLOT}
        R.~Brun and N.C.~Somon,
        ``HPLOT User Guide'',
        CERN Program Library Y251 (1988).
    %
    \bibitem{HIGZ}
        R.~Bock, R.~Brun, O.~Couet, N.C.~Somon, C.E. Vandoni and P. Zanarini,
        ``HIGZ User Guide'',
        CERN Program Library Q120.
    %
    \bibitem{Gilbert78}
        G.~Guignard,
        ``A general treatment of resonances in accelerators'',
        CERN 78--11 (1978).
    %
    \bibitem{Berz89}
        M.~Berz,
        ``Differential algebra description of beam dynamics to very high orders'',
        Particle Accelerators, 1989, Vol. \underline{24}, pp. 109--124.
    %
    \bibitem{DAFOR}
        M.~Berz,
        ``DAFOR -- Differential Algebra Precompiler Version 3, Reference Manual'',
        MSUCL--755 (1991).
    %
    \bibitem{Sixvec}
        F.~Schmidt and M.~Vaenttinen,
        ``Vectorisation of the single particle tracking program SixTrack'',
        CERN SL Note 90--20 (1990) (AP).
    %
    \bibitem{thesis}
        F.~Schmidt,
        ``Untersuchungen zur dynamischen Akzeptanz von Protonenbeschleunigern und ihre Begrenzung durch chaotische Bewegung'',
        DESY HERA 88--02, (1988).
    %
    %\bibitem{Daspeed} J.~Irwin, private communication.
    %
    \bibitem{CONVERTOR}
        H.~Grote,
        ``A MAD--SixTrack interface'',
        SL Note 97--02 (AP).
    %
    \bibitem{sixphys}
        SixTrack Physics Manual,
        \url{http://sixtrack.web.cern.ch/SixTrack/}
    %
    \bibitem{h5_doc}
        HDF5 Software Documentation,
        \url{https://support.hdfgroup.org/HDF5/doc/H}
    %
    \bibitem{Forest89}
        M.~Berz, \'{E}. Forest and J. Irwin,
        ``Normal form methods for complicated periodic systems: a complete solution using differential algebra and lie operators'',
        Particle Accelerators, 1989, Vol. \underline{24}, pp. 91--107.
    %
    \bibitem{BasErs}
        M.~Bassetti and G.A.~Erskine,
        ``Closed expression for the electrical field of a two-dimensional Gaussian charge'',
        CERN--ISR--TH/80--06.
    %
    \bibitem{Hirata}
        K.~Hirata, H.~Moshammer, F.~Ruggiero and M.~Bassetti,
        ``Synchro-Beam interaction'',
        CERN SL-AP/90-02 (1990) and Proc. Workshop on Beam Dynamics Issues of High-Luminosity Asymmetric Collider Rings, Berkeley, 1990, ed. A.M. Sessler (AIP Conf. Proc. 214, New York, 1990), pp. 389-404;\\
        K.~Hirata, H.~Moshammer and F.~Ruggiero,
        ``A symplectic beam-beam interaction with energy change'',
        KEK preprint 92-117 A (1992) and Part. Accel. 40, 205-228 (1993);\\
        K.~Hirata,
        ``BBC User's Guide; A Computer Code for Beam-Beam Interaction with a Crossing Angle, version 3.4'',
        SL-Note 97-57 AP.
    %
    \bibitem{ripbeam}
        L.H.A.~Leunissen, F.~Schmidt and G.~Ripken,
        ``6D Beam--Beam Kick including Coupled Motion'',
        LHC Project Report 369, \url{http://cern.ch/Frank.Schmidt/report/ripken\_new.pdf}.
    %
    \bibitem{SODD}
        F.~Schmidt,
        ``SODD: A Computer Code to calculate Detuning and Distortion Function Terms in First and Second Order'',
        CERN SL/Note 99--009 (AP), \url{http://cern.ch/Frank.Schmidt/report/sodd\_manual.pdf}
    %
    \bibitem{MAD}
        H.~Grote and F.C.~Iselin,
        ``The MAD program (Methodical Accelerator Design), Version 8.10, User's Reference Manual'',
        CERN SL 90--13 (AP) (Rev. 4), \url{http://cern.ch/Hans.Grote/mad/mad8/doc/mad8\_user.ps.gz}.
    %
    \bibitem{dipedge}
        R.~Molloy and S.~Blitz,
        ``Fringe Field Effects on Bending Magnets, Derived for, TRANSPORT/TURTLE'',
        FERMILAB-TM-2564-AD-APC-PPD, \url{http://lss.fnal.gov/archive/test-tm/2000/fermilab-tm-2564-ad-apc-ppd.pdf}
    %
    \bibitem{ERIC} private communication. % with whom?
    %
    \bibitem{RANECU}
        F.~James,
        ``A Review of Pseudorandom Number Generators'',
        CERN DD/ 88/22, 1988.
        % F.~James,
        % ``A review of pseudo-random number generators'',
        % to be published in Computer Physics Communication.
    %
    \bibitem{Auti}
        B.~Autin and Y.~Marti,
        ``Closed Orbit Correction of A.G. Machines Using a Small Number of Magnets'',
        CERN--ISR--MA/73--17.
    %
    \bibitem{Massimo}
        M.~Giovannozzi,
        ``Description of software tools to perform tune-shift correction using normal forms'',
        CERN SL Note 93--111 (AP).
    %
    \bibitem{Refine}
        F.~Schmidt, F.~Willeke and F.~Zimmermann,
        ``Comparison of methods to determine long-term stability in proton storage rings'',
        1991, Particle Accelerators, Vol. \underline{35}, pp. 249--256.
    %
    \bibitem{plato1}
        R.~Bartolini, A.~Bazzani, M.~Giovannozzi, W.~Scandale, E.~Todesco,
        ``Tune evaluation in simulations and experiments'',
        Part. Accel. 52 147
    %
    \bibitem{plato2}
        M.~Giovannozzi, E.~Todesco, A.~Bazzani and R.~Bartolini (1997),
        ``PLATO: a program library for the analysis of nonlinear betatronic motion'',
        Nucl. Instrum. and Methods A 388 1
    %
    \bibitem{NAFFpaper}
        J.~Laskar, C.~Froeschle and C.~Celletti,
        ``The measure of chaos by the numerical analysis of the fundamental frequencies. Application to the standard mapping'',
        Physica D, vol. 56, pp 253-269, 1992.
    %
    \bibitem{NAFFpaper2}
        S.~Kostoglou, N.~Karastathis, Y.~Papaphilippou, D.~Pellegrini and P.~Zisopoulos,
        ``Development of computational tools for noise studies in the LHC'',
        2017, Proceedings of IPAC'17, Copenhagen, Denmark, 2017.
    %
    \bibitem{sixbuild}
        SixTrack build manual, see SixTrack website, \url{http://sixtrack.web.cern.ch/SixTrack/}
    %
    \bibitem{sixdesk1}
        SixDesk manual, see SixTrack website, \url{http://sixtrack.web.cern.ch/SixTrack/}
    %
    \bibitem{sixdesk2}
        SixDesk manual, \url{https://www.overleaf.com/1345694dwypbp#/3325092/}
    %
    % \bibitem{PATRAC}
    %     A.~Hilaire and A.~Warman,
    %     ``A program for single particle tracking'',
    %     CERN SPS/88--8 (AMS).
    %
    \bibitem{RFmultsPaper}
        J.~B.~Garcia et al.,
        ``Long term dynamics of the high luminosity Large Hadron Collider with crab cavities'',
        2016, PHYSICAL REVIEW ACCELERATORS AND BEAMS 19, 101003 (2016).
    %
    \bibitem{DYNKpaper}
        K.~Sjobak, H.~Burkhardt, R.D.~Maria, A.~Mereghetti and A.~Santamaria,
        ``General functionality for turn-dependent element properties in SixTrack'',
        2015, Procedings of IPAC'13, Richmond, VA, USA, May 2015.
    %
    \bibitem{DYNKpaper2}
        K.~Sjobak, V.K.~Berglyd~Olsen, R.~De~Maria, M.~Fitterer, A.~Santamaría~García, H. Garcia-Morales, A.~Mereghetti, J.F.~Wagner, S.J.~Wretborn,
        ``Dynamic simulations in SixTrack'',
        CERN
    %
    \bibitem{SRussen:fieldComp}
        S.~Russenschuck,
        ``Field computation for Accelerator Magnets'',
        Wiley-VCH, 2010
    %
    \bibitem{BurlaKing:CurrentRamp}
        P.~Burla, Q.~King and J.G.~Pett,
        ``Optimisation of the current ramp for the LHC'',
        Proceedings of the 1999 Particle Accelerator Conference, New York, 1999.
    %
    \bibitem{collimat:trenkler}
        T.~Trenkler, J.B.~Jeanneret,
        ``K2, A software package evaluating collimation systems in circular colliders (manual)'',
        CERN SL/94–105 (AP), 1994.
    %
    \bibitem{collimat:robert_demolaize}
        G.~Robert-Demolaize, R.~Assmann, S.~Redaelli, F.~Schmidt,
        ``A new version of SixtTrack with collimation and aperture interface'',
        CERN, Geneva, Switzerland (PAC 2005).
    %
    \bibitem{collimat:assmann}
        R.~Assmann, J.B.~Jeanneret, D.~Kaltchev,
        ``Status of Robustness Studies for the LHC Collimation'',
        APAC 2001.
    %
    \bibitem{pythia8}
        T.~Sjöstrand, S.~Mrenna, P.~Skands,
        ``A Brief Introduction to PYTHIA 8.1 '',
        Comput. Phys. Comm. 178 (2008) 852 [arXiv:0710.3820].
    %
    \bibitem{pythia_phys}
        T.~Sjöstrand, S.~Mrenna, P.~Skands,
        ``PYTHIA 6.4 Physics and Manual'',
        JHEP05 (2006) 026.
    %
    \bibitem{coupling:1}
        V.~Vlachoudis {\it et al.}, ``Status of Fluka coupling to Sixtrack'',
        Proceedings of Tracking for Collimation Workshop (WP5), CERN, Geneva,
        Switzerland (in publication).
    %
    \bibitem{coupling:2}
        A.~Mereghetti, Performance Evaluation of the SPS Scraping System
        in View of the High Luminosity LHC, Ph.~D.~thesis, UniMAN, Manchester,
        UK (2015).
    %
    \bibitem{ffield:1}
        B. Dalena \emph{et al.}, ``Fringe Field Modeling for the High Luminosity LHC Large Aperture quadrupole'',
        Proceedings of IPAC’14, Dresden, Germany, June 2014, paper TUPRO002, pp.~993--996.
    %
    \bibitem{ffield:2}
        T. Pugnat \emph{et al.}, ``Accurate and Efficient Tracking in Electromagnetic Quadrupoles'',
        Proceedings of IPAC’18, Vancouver, Canada, June 2014, paper THPAK004, pp.~3207.
    %
    \bibitem{ffield:3}
        A. Simona \emph{et al.}, ``High order time integrators for the simulation of charged particle motion in magnetic quadrupoles'',
        Elsevier, February 2019.
    %
    \bibitem{ranlux}
        F. James, ``RANLUX: A FORTRAN Implementation of the High Quality Pseudorandom Number Generator of Luscher''.
        Comput.Phys.Commun. 79 (1994): 111–14.DOI:10.1016/0010-4655(94)90233-X.
    %
    \bibitem{ranecu1}
        F. James, ``A Review of Pseudorandom Number Generators'',
        Computer Physics Communications 60, no. 3 (1 October 1990): 329–44. DOI:10.1016/0010-4655(90)90032-V.
    %
    \bibitem{ranecu2}
        P. L’Ecuyer, ``Efficient and Portable Combined Random Number Generators'',
        Commun. ACM 31, no. 6 (June 1988): 742–751. DOI:10.1145/62959.62969.
    %
    \bibitem{crystal:1}
    V.~Previtali, ``Performance Evaluation of a Crystal-enhanced Collimation System for the LHC'',
    Ph.~D.~thesis, EPFL, Lausanne, Switzerland (2010).
    %
    \bibitem{crystal:2}
    D.~Mirarchi, ``Crystal collimation for LHC'',
    Ph.~D.~thesis, Imperial College, London, UK (2015).
    %
    \bibitem{crystal:3}
    D.~Mirarchi \emph{et al.}, ``A crystal routine for collimation studies in circular proton accelerators'',
    Proceedings of the 6$^{\mathrm th}$ International Conference Channeling: ``Charged \& Neutral Particles Channeling Phenomena'', Capri, Italy (2014).
    %
    \bibitem{crystal:4}
    D.~Mirarchi \emph{et al.}, ``Crystal implementation in SixTrack for proton beams'',
    ICFA Mini-Workshop on Tracking for Collimation in Particle Accelerators, CERN, Geneva, Switzerland (2015).
    %
    \bibitem{crystal:5}
    F.~Forcher, ``An improved simulation routine for modelling coherent high-energy proton interactions with bent crystals'',
    Bachelor~thesis, UniPD, Padova, Italy (2017).
    %
    \bibitem{crystal:6}
    R.~Rossi, ``Experimental Assessment of Crystal Collimation at the Large Hadron Collider'',
    Ph.~D.~thesis, Universit\`a La Sapienza', Roma, Italy (2018).
    %
    \addcontentsline{toc}{chapter}{Bibliography}
\end{thebibliography}
