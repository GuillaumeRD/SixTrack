
\chapter{Machine Geometry} \label{MaGe}

% ================================================================================================================================ %
\section{Single Elements} \label{SinEle}

The \textit{Single Elements} input block defines the name and type of linear and non-linear elements, the inverse bending radius or multipole strength respectively, and the strength and length of the elements.\index{single elements}\index{SING}
Linear and non-linear elements are distinguished by length -- linear elements have a non-zero length and non-linear elements have zero length.
Both kinds of elements can appear in the input block in arbitrary order.
The input line has a different format for linear and non-linear elements.
Moreover, the multipoles, being a set of non-linear elements, are treated in a special way.
The maximum number of elements is set as a parameter (see Appendix~\ref{DSP}).

\bigskip
\begin{tabular}{@{}lp{0.8\linewidth}}
    \textbf{Keyword}    & \texttt{SING} \\
    \textbf{Data lines} & Variable \\
    \textbf{Format}     & Described in the following sections.
\end{tabular}

% ================================================================================================================================ %
\subsection{Linear Elements} \label{LinEle}

Each linear single element has a name, type, inverse bending radius, focusing and a non-zero length.\index{linear element}

\paragraph{Format} \texttt{name type $\varrho^{-1}$ K length}

\bigskip
\begin{tabular}{@{}lp{0.8\linewidth}}
    \texttt{name} & May contain up to 48 characters. \\
    \texttt{type} & As shown in the table~\ref{T-LinEle} .\\
    \texttt{$ \varrho^{-1}$} &  Inverse bending radius in $\mathrm{m}^{-1}$. \\
    \texttt{K} & Focusing strength in $\mathrm{m}^{-2}$. \\
    \texttt{length} & Magnet length in meters.
\end{tabular}

\begin{table}[h]
    \caption{Different Types of Linear Elements}
    \label{T-LinEle}
    \centering
    \renewcommand{\arraystretch}{1.5}
    \begin{tabular}{|c|c|c|l|}
        \hline
        \rowcolor{blue!30}
        \textbf{Type} & $\mathbf{\varrho^{-1}}$ & $\mathbf{K}$ & \textbf{Description} \\
        \hline
        0 & 0 & 0 & Drift length magnet \index{drift length}\\
        \hline
        1 & X & 0 & Horizontal (rectangular) bending \index{bending magnet}\\
        \hline
        2 & 0 & X & Quadrupole (-- focusing, + defocusing) \\
        \hline
        3 & X & 0 & Horizontal (sector) bending \\
        \hline
        4 & X & 0 & Vertical (rectangular) bending \\
        \hline
        5 & X & 0 & Vertical (sector) bending \\
        \hline
        6 & X & X & Horizontal combined function magnet \\
        \hline
        7 & X & X & Vertical combined function magnet \\
        \hline
        8 & X & 0 & Edge focusing \index{edge focusing}\\
        \hline
    \end{tabular}
\end{table}

\paragraph{Remarks}~
\begin{enumerate}
    \item For the horizontal plane the bending radius is defined to be negative \mbox{($\varrho < 0$).} This is different from other programs like MAD-X~\cite{MAD}.
    \item $K < 0 $ corresponds to a horizontal focusing quadrupole.\index{quadrupole}
    \item For the length of an edge focusing element (type=8) the same value must be used as for the corresponding bending magnet. A sector bending magnet is transformed into a rectangular magnet with an edge focusing element of positive length on either side, while for the opposite transformation a negative length is required.
    \item It is important to note that the splitting of a rectangular magnet\index{rectangular magnet}, which is sometimes necessary if multipole errors are to be introduced, does change the linear optics. It is therefore advisable to replace the rectangular magnet with a sector magnet, which can be split without affecting the linear optics, and make an overall transformation into a rectangular magnet via edge focusing elements. Do not forget to use the total length of dipole as the length of the edge focusing element.
\end{enumerate}

% ================================================================================================================================ %
\subsection{Non-linear Elements} \label{NonEle}

\paragraph{Format} \texttt{name type $K_{n}$-strength rms-strength length}\index{non-linear elements}

\bigskip
\begin{tabular}{@{}lp{0.8\linewidth}}
    \texttt{name} & May contain up to 48 characters. \\
    \texttt{type} & As shown in table~\ref{T-NonEle}. \\
    \texttt{$K_{n}$-strength} & Average multipole strength. \\
    \texttt{rms-strength} & Random multipole strength. \\
    \texttt{length} & Must be 0.
\end{tabular}

\begin{table}[h]
    \caption{Different Types of Non-linear Elements}
    \label{T-NonEle}
    \centering
    \renewcommand{\arraystretch}{1.5}
    \begin{tabular}{|c|l|l|}
        \hline
        \rowcolor{blue!30}
        \textbf{Type} & \textbf{Strength} & \textbf{Description} \\
        \hline
        \texttt{~0} & -- & Observation point (for instance for aperture limitations). \\
        \hline
        \texttt{~1} & $b_{1} [\mathrm{rad} \cdot \mathrm{m}^{0}]$ & Horizontal bending kick. \\
        \texttt{-1} & $a_{1}$ & Vertical bending kick.\index{bending kick} \\
        \hline
        \texttt{~2} & $b_{2} [\mathrm{rad} \cdot \mathrm{m}^{-1}]$ & Normal quadrupole kick.\index{quadrupole kick} \\
        \texttt{-2} & $a_{2}$ & Skew quadrupole kick.\index{skew quadrupole kick} \\
        \hline
        \vdots & & \\
        \hline
        \texttt{~10} & $b_{10} [\mathrm{rad} \cdot \mathrm{m}^{-9}]$ & Normal $20^{th}$ pole.\index{n-th pole magnet} \\
        \texttt{-10} & $a_{10}$ & Skew $20^{th}$ pole.\index{skew n-th pole magnet} \\
        \hline
    \end{tabular}
\end{table}

\paragraph{Remarks}~
\begin{enumerate}
    \item Because the horizontal bending magnet is defined to have a negative bending radius, the sign for normal elements is different from other programs like MAD-X, while skew elements have the same sign.
    \item Again contrary to other programs the factor \mbox{$(n-1)$!} is already included in the multipole strength, which is defined as follows:
    \begin{itemize}
        \item for normal elements:\index{normal element}
        \begin{equation*}
            b_{n}(\mathrm{SixTrack}) = \frac{-1}{(n-1)!}\,L_{\mathrm{element}}\,b_{n}(\mathrm{MAD})
        \end{equation*}
        \item for skew elements:\index{skew element}
        \begin{equation*}
            a_{n}(\mathrm{SixTrack}) = \frac{1}{(n-1)!}\,L_{\mathrm{element}}\,a_{n}(\mathrm{MAD})
        \end{equation*}
    \end{itemize}
    \item Unlike in RACETRACK\index{RACETRACK}, the horizontal and vertical displacements do not fit into the 80 character input lines of SixTrack\@. They have to be introduced in a separate \textit{Displacements of Elements} input block (\ref{DisEle}).
\end{enumerate}

% ================================================================================================================================ %
\subsection{Multipole Blocks} \label{MulBlo}

A set of normal, normal-r.m.s., skew, and skew-r.m.s. errors can be combined effectively.\index{multipole coefficient}\index{multipole}
The actual values for the strengths have to be given in a separate \textit{Multipole Coefficient} input block (\ref{MulCoe}) which must have the same name.
To consider the curvature of dipoles which are replaced by drifts and dipole kicks this block is used in two different ways.

\paragraph{Format} \texttt{name type cstr cref length}

\begin{description}
    \item [Marker for high order kick (default)]~

    \bigskip
    \begin{tabular}{@{}lp{0.8\linewidth}}
        \texttt{name} & May contain up to 48 characters. \\
        \texttt{type} & Must be 11. \\
        \texttt{cstr} & The bending strength given in the \textit{Multipole Coefficient} input block (\ref{MulCoe}) is multiplied with this factor.\\
        \texttt{cref} & The reference radius given in the \textit{Multipole Coefficient} input block (\ref{MulCoe}) will be multiplied by this factor. If it is zero the multipole block will be ignored.\\
        \texttt{length} & Must be 0.
    \end{tabular}
    \item [Default + dipole curvature]~

    \bigskip
    \begin{tabular}{@{}lp{0.8\linewidth}}
        \texttt{name} & May contain up to 48 characters. \\
        \texttt{type} & Must be 11. \\
        \texttt{cstr} & The bending strength [rad] of horizontal or vertical dipole. Internally the value is set to one to allow the processing of a multipole block (\ref{MulCoe}). \\
        \texttt{cref} & The length [m] of the dipole that is approximated by a kick. Internally this value is set to one to allow the processing of a multipole block (\ref{MulCoe}). \\
        \texttt{length} & length = -1: horizontal dipole. \\
                        & length = -2: vertical dipole.
    \end{tabular}
\end{description}

\paragraph{Remarks}
The definition of the multipole strength in a block will be given in (\ref{MulCoe}).

% ================================================================================================================================ %
\subsection{Cavities} \label{Cavities}

\paragraph{Format} \texttt{name type u0 harm lag}\index{cavities}

\bigskip
\begin{tabular}{@{}lp{0.8\linewidth}}
    \texttt{name} & May contain up to 48 characters. \\
    \texttt{type} & Type identifier is $+12$ and $-12$ for above and below transition energy respectively. \\
    \texttt{u0}   & Circumference voltage in [MV]. \\
    \texttt{harm} & Harmonic number. \\
    \texttt{lag}  & Lag angle [degrees] in the cavity (zero is default).
\end{tabular}

% ================================================================================================================================ %
\subsection{Beam--Beam Lens} \label{BBS}

Depending on the setting in the \texttt{BEAM} block of \texttt{fort.3} (Section~\ref{BeamBeam}), there are two ways to define a beam beam lens in the SINGLE ELEMENTS list.\index{beam--beam}

\bigskip
\noindent\textbf{When the \texttt{EXPERT} flag is set in the \texttt{BEAM} block:}
The parameters of the beam--beam lens is defined there.
In this case, only the element name and type and the location within the lattice remain in the \texttt{fort.2} element definition.

\paragraph{Format} \texttt{name type 0 0 0 0 0 0}

\bigskip
\begin{tabular}{@{}lp{0.8\linewidth}}
    \texttt{name} & May contain up to 48 characters. \\
    \texttt{type} & 20 \\
    \texttt{}     & The rest of the parameters are ignored and should be set to zero.
\end{tabular}

\bigskip
\noindent\textbf{When the \texttt{EXPERT} flag is not set:}
The ``traditional'' format is used.

\paragraph{Format} \texttt{name type h-sep v-sep strength-ratio ${\sigma\_\rm{hor}}^2$ ${\sigma\_\rm{ver}}^2$ ${\sigma\_\rm{lon}}^2$}

\bigskip
\begin{longtabu}{@{}lp{0.7\linewidth}}
    \texttt{name}  & May contain up to 48 characters. \\
    \texttt{type}  & 20 \\
    \texttt{h-sep} & Horizontal beam--beam separation [mm]. \\
    \texttt{v-sep} & Vertical beam--beam separation [mm]. \\
    \texttt{strength-ratio} & Strength ratio with respect to the nominal beam--beam kick strength. This is useful, in particular for 4D, to allow for splitting one beam--beam kick into several (longitudinally close by) kicks. \\
    \texttt{${\sigma\_\rm{hor}}^2$} & When the flag $lhc=2$ is set in the \texttt{BEAM} block of the \texttt{fort.3} file, this column represent the horizontal $\sigma$ for the strong beam $\rm{[mm^2]}$. \\
    \texttt{${\sigma\_\rm{ver}}^2$} & When the flag $lhc=2$ is set in the \texttt{BEAM} block of the \texttt{fort.3} file, this column represent the vertical $\sigma$ for the strong beam $\rm{[mm^2]}$. \\
    \texttt{${\sigma\_\rm{lon}}^2$} & This variable is for future purposes, at the present it is always equal to zero.
\end{longtabu}

\paragraph{Remarks}
These beam--beam elements become active when the ``Beam--Beam'' input block~\ref{BeamBeam} is used.

% ================================================================================================================================ %
\subsection{Wire} \label{WIRE}

\paragraph{Format} \texttt{name type}\index{wire element}

\bigskip
\begin{tabular}{@{}lp{0.8\linewidth}}
    \texttt{name} & May contain up to 48 characters. \\
    \texttt{type} & 15
\end{tabular}

\paragraph{Remarks}
The ``wire'' elements become active when the \texttt{WIRE} input block~\ref{sec:WIRE} is used.
All parameters except name and type have to be set to zero, otherwise SixTrack aborts.
The parameters for the wire are defined in the \texttt{WIRE} input block.

% ================================================================================================================================ %
\subsection{``Phase-trombone'' or Matrix Element} \label{PT}

\paragraph{Format} \texttt{name type}\index{phase trombone}

\bigskip
\begin{tabular}{@{}lp{0.8\linewidth}}
    \texttt{name} & May contain up to 48 characters \\
    \texttt{type} & 22
\end{tabular}

\paragraph{Remarks}
These ``trombone'' elements become active when the ``Phase Trombone Element'' input block~\ref{Trombone} is used.

% ================================================================================================================================ %
\subsection{AC Dipole} \label{ACDIP}

\paragraph{Format} \texttt{name type ACdipAmp Qd ACdipPhase}\index{AC dipole}

\bigskip
\begin{tabular}{@{}lp{0.8\linewidth}}
    \texttt{name} & May contain up to 48 characters. \\
    \texttt{type} & Type identifier is $+16$ and $-16$ for horizontal and vertical AC dipoles respectively. \\
    \texttt{ACdipAmp} & Maximum excitation amplitude [Tm]. \\
    \texttt{Qd}   & Excitation frequency in units of [$2 \times \pi$]. \\
    \texttt{ACdipPhase} & Phase of the harmonic excitation in radians.
\end{tabular}

\paragraph{Remarks}
The length of the ramps and the flat top are specified in the ``Displacement'' block~\ref{DisEle}. The energy introduced in the ``Initial coordinates'' block~\ref{IniCoo} is used to compute the deflection angle.

% ================================================================================================================================ %
\subsection{Dipole Edge}

\paragraph{Format} \texttt{name type $r_{21}$ $r_{43}$}\index{dipole edge}\index{dipedge}

\bigskip
\begin{tabular}{@{}lp{0.8\linewidth}}
    \texttt{name} & May contain up to 48 characters. \\
    \texttt{type} & 24 \\
    \texttt{$r_{21}$} & Horizontal edge focusing. \\
    \texttt{$r_{43}$} & Vertical edge focusing.
\end{tabular}

\paragraph{Remarks}
MAD-X is outputting the correct format when using the dipedge element. An example of  the hard edge model is described in the physics guide \cite{sixphys}, which gives $r_{21} = -r_{43}$.
Note that the values of the vertical edge focusing is dependent on the modeling of the fringe fields \cite{dipedge}.
A particle with position $x_{1},y_1$ and angle $x'_{1},y'_1$ will have the angle $x'_{2},y'_2$ after passing through the dipedge element.
The following equations describe their relation:
\begin{eqnarray}
    x'_{2} = x'_{1} + x_{1}\frac{r_{21}}{1+\delta} \\
    y'_{2} = y'_{1} + y_{1}\frac{r_{43}}{1+\delta}
\end{eqnarray}

% This element needs more documentation and possibly testing -- not currently supported!

% ================================================================================================================================ %
% \subsection{Solenoid}

% \paragraph{Format} {\em name type ed ek}
% \begin{description}
% \item[name] May contain up to sixteen characters
% \item[type] Type identifier $25$
% \item[ed] First argument, meaning unknown
% \item[ek] Second argument, meaning unknown
% \end{description}

% This element needs more documentation and possibly testing -- not currently supported! There are also a few things in the code which looks very strange.

% ================================================================================================================================ %
\subsection{Crab Cavity} \label{CrabCav}

\paragraph{Format} \texttt{name type voltage frequency phase}\index{crab cavity}

\bigskip
\begin{tabular}{@{}lp{0.8\linewidth}}
    \texttt{name} & May contain up to 48 characters. \\
    \texttt{type} & Type identifier is $+23$ and $-23$ for horizontal and vertical crab cavities respectively. \\
    \texttt{voltage} & Crab Cavity voltage [MV]. \\
    \texttt{frequency} & Crab Cavity frequency [MHz]. \\
    \texttt{phase} & Phase of the excitation in radians.
\end{tabular}

\paragraph{Remarks}~\\

\noindent How to use the crab cavity from MAD-X (using rfmultipole) to SixTrack:\\
\bigskip
\noindent In the MAD-X script write:
\begin{cverbatim}
MULT.1, FREQ=<freq in MHz>., KNL=\{V [MV]/E0[MeV]\}, PNL=\{phase\}, TILT=<H: 0; V:PI/2.>;
\end{cverbatim}
where phase is $0.25$ (phase for multipoles in SixTrack).
As an example, to have the effect of a vertical Crab Cavity of $f=400~\mathrm{MHz}$, $V=6~\mathrm{MV}$, beam energy [MeV]: \texttt{BEAM -> PC/1e3}, use the following line:\\
\begin{cverbatim}
MULT.1, FREQ=400., KNL={6./BEAM -> PC/1e3}, PNL={0.25}, TILT=PI/2.;
\end{cverbatim}
This creates the following line in \texttt{fort.2}:
\begin{cverbatim}
mult.1d  -23  6.00000000e+00  4.00000000e+02  0.00000000e+00  0.00000000e+00  0.00000000e+00  0.00000000e+00
\end{cverbatim}
If you don’t want to have a vertical Crab Cavity then just remove the \texttt{TILT}.
If you don’t want to have CC but a simple dipole field then remove the \texttt{FREQ} parameter.

% ================================================================================================================================ %
\subsection{RF Multipole}

Provides a kick in the form of\index{RF multipole}
\begin{align}
    \Delta x'+i\Delta y' =& \frac{k}{1+\delta} (x+iy)^n \cos (\phi - 2 \pi f t) \\
    \Delta \delta =& P_0 \frac{k}{1+\delta} \frac{(x+iy)^{n+1}}{(n+1)!} \cos (\phi - 2 \pi f t)
\end{align}

\paragraph{Format} \texttt{name type name kick frequency phase}
\todo[inline]{Check the phase keyword}

\bigskip
\begin{tabular}{@{}lp{0.8\linewidth}}
    \texttt{name} & May contain up to 48 characters. \\
    \texttt{type} & 26: normal quadrupole, -26 skew quadrupole,\index{quadrupole}\index{skew quadrupole} \\
                  & 27: normal sextupole, -27 skew sextupole,\index{sextupole}\index{skew sextupole} \\
                  & 28: normal octupole, -28 skew octupole.\index{ocyupole}\index{skew octupole} \\
    \texttt{kick} & maximum normalized kick $k$. \\
    \texttt{frequency} & frequency $f$ in [MHz].
\end{tabular}

\paragraph{Remarks}~\\
How to use the RF multipoles (from MAD-X to SixTrack):\\

\noindent\textbf{2\textsuperscript{nd} order multipole (quadrupole):}\index{quadrupole}\\
\noindent In the MAD-X script write:
\begin{ctverbatim}
MULT.1, KNL=\{0,-0.06*1e-3\}, PNL=\{0, 0.25\};
\end{ctverbatim}
where \texttt{-0.06*1e-3} is the $b_2$ value in units of $\mathrm{Tm/m}^{n-1}$.\\
This gives the following single element in \texttt{fort.2}:
\begin{ctverbatim}
mult.1q  26  6.00000000e-05  400.00000000e+00  -1.570796327e+00  0.00000000e+00  0.00000000e+00  0.00000000e+00
\end{ctverbatim}

\noindent\textbf{3\textsuperscript{rd} order multipole (sextupole):}\index{sextupole}\\
\noindent In the MAD-X script write:
\begin{ctverbatim}
MULT.1, FREQ=400., KNL=\{0,0,1159.*1e-3\}, PNL=\{0,0,0.25\};
\end{ctverbatim}
where \texttt{1159.*1e-3} is the $b_3$ value in units of $\mathrm{Tm/m}^{n-1}$.\\
This gives the following single element in \texttt{fort.2}:
\begin{ctverbatim}
mult.1s  27 -5.79500000e-01  4.00000000e+02  -1.570796327e+00  0.00000000e+00  0.00000000e+00  0.00000000e+00
\end{ctverbatim}

\noindent\textbf{4\textsuperscript{th} order multipole (octupole):}\index{octupole}\\
\noindent In  the MAD-X script write:
\begin{ctverbatim}
MULT.1, FREQ=400., KNL=\{0,0,0,-4.*1e-3\}, PNL=\{0,0,0,0.25\};
\end{ctverbatim}
where \texttt{-4.*1e-3} is the $b_4$ value in units of $\mathrm{Tm/m}^{n-1}$.\\
This gives the following single element in \texttt{fort.2}:
\begin{ctverbatim}
mult.1o  28  6.666666667e-04  4.00000000e+02  -1.570796327e+00  0.00000000e+00  0.00000000e+00  0.00000000e+00
\end{ctverbatim}

\bigskip
\noindent The values of $b_2$, $b_3$, and $b_4$ used in the above examples were taken from Table II of paper~\cite{RFmultsPaper}.

The effect of these multipoles was checked on a beam of particles with $x=x^{\prime}=y^{\prime}=0$, and $y= 1, 2, \text{ and } 3~\mathrm{mm}$, with different $z$ positions.
The effect on $y'$ was linear, quadratic and cubic with $y$ when using $b_2$, $b_3$, or $b_4$, respectively, as expected.
Furthermore, the amplitude of the $y'$ agrees with the analytical formulas found in the appendix of this paper~\cite{RFmultsPaper} under ``Normal quadrupole/sextupole/octupole''.

\textit{Important note:} $B\rho$ and the factorial $(n-1)!$ are already included in K2, K3 etc of MAD-X, i.e.\ $b_3=1159\cdot10^{-3}$ in MAD-X results in a kick as if $b_3$ is $1159\cdot10^{-3}/(n-1)!$.
So in order for this paper's~\cite{RFmultsPaper} analytical equations to be compatible with MAD-X, the equations for normal quadrupole should read as
\begin{equation*}
    \Delta x'=-\frac{b_2}{(2-1)! ~ B\rho} \ldots~.
\end{equation*}

% ================================================================================================================================ %
\subsection{Electron Lens} \label{ELEN}

\paragraph{Format} \texttt{name type}\index{electron lens}\index{ELEN}\index{ELENS}

\bigskip
\begin{tabular}{@{}lp{0.8\linewidth}}
    \texttt{name} & May contain up to 48 characters. \\
    \texttt{type} & 29
\end{tabular}

\paragraph{Remarks}
The ``e-lens'' elements become active when the \texttt{ELEN} input block~\ref{sec:elen} is used.
All parameters except name and type have to be set to zero in the list of single elements, otherwise SixTrack aborts.
The parameters for the e-lens are defined in the \texttt{ELEN} input block.

% ================================================================================================================================ %
\subsection{Scattering point} \label{SCAT}

\paragraph{Format} \texttt{name type}\index{SCAT}\index{elastic scattering}

\bigskip
\begin{tabular}{@{}lp{0.8\linewidth}}
    \texttt{name} & May contain up to 48 characters. \\
    \texttt{type} & 40
\end{tabular}

\paragraph{Remarks}
The ``scattering'' elements become active when the \texttt{SCAT}ter input block~\ref{sec:scatter} is used.
All parameters except name and type have to be set to zero in the list of single elements, otherwise SixTrack aborts.
The parameters of the scattering are defined in the \texttt{SCAT}ter input block.

% ================================================================================================================================ %
\subsection{Beam Position Monitor} \label{BPM}

\paragraph{Format} \texttt{BPMname 0 0 0 0}\index{beam position monitor}\index{BPM}

\bigskip
\begin{tabular}{@{}lp{0.8\linewidth}}
    \texttt{BPMname} & Must start with ``BP'' and maybe followed by 46 characters.
\end{tabular}

\paragraph{Remarks}
This element dumps the coordinates of the 1st particle to the file with name ``BPMname''.
The file contains 7 columns: $x$,$x'$, $y$,$y'$, $ct$,$\delta p/p$ and $E$.
Usual SixTrack units are used.
Any number of \texttt{BPM} elements can be used but the names must differ.

% ================================================================================================================================ %
\subsection{Other Element Types}

Some other elements, such as dipole edge (24), solenoid (25), multipole RF kicks ($\pm$26, $\pm$27, $\pm$28) are accepted by SixTrack, but they are not currently supported by the development team or tested for correctness.\index{dipedge}\index{solenoid}
It is therefore advised to not use these elements.

% ================================================================================================================================ %
\section{Block Definitions} \label{BloDef}

In four-dimensional transverse tracking, the linear elements between non-linear elements can be combined to a single linear block to save computing time.\index{block definitions}\index{BLOC}

\bigskip
\begin{tabular}{@{}lp{0.8\linewidth}}
    \textbf{Keyword}    & \texttt{BLOC} \\
    \textbf{Data lines} & $>1$ \\
    \textbf{Format}     & First line: \texttt{mper msym(1) \dots~msym(mper)} (integers) \\
                        & From second line: \texttt{block-name \{element-name\}}
\end{tabular}

\paragraph{Format Description}~

\bigskip
\begin{longtabu}{@{}lp{0.75\linewidth}}
    \texttt{mper} & Number of super periods\index{super periods}. The following set of blocks is considered a \texttt{super-period}. The accelerator consists of \texttt{mper} super-periods. \\
    \texttt{msym(i)} & $\pm$ 1 for each super period. If $\mathrm{msym}(i)=1$, the \mbox{$i$'th} super period will be built up in the order in which linear elements appear in the blocks  below. If $\mathrm{msym}(i)=-1$, the super period will be built up in reverse order. \\
    \texttt{block-name} & The name of the block with up to 48 characters. \\
    \texttt{element-name} & The element names have to appear as a linear element in the list of ``single elements'' (\ref{LinEle}). If one line is too short to contain all the elements of a block, a line with additional elements to the same block can be added. At least 5 (five) blanks must appear at the beginning of the extra line so that names of blocks and names of linear elements in a block can be distinguished.
\end{longtabu}

\paragraph{Remarks}
\begin{enumerate}
    \item When synchrotron oscillation\index{synchrotron oscillation} is introduced, the linear elements can no longer be lumped into one block, because in that case even a drift length magnet is a non-linear element with respect to the longitudinal plane. However, the block structure is still kept to make use of the speed-up in case one can restrict the studies to the four-dimensional case.
    \item The maximum number of blocks and the maximum number of entries in each block are defined as parameters (Appendix~\ref{DSP}).
    \item The inversion of a super-period ($\mathrm{msym(i)} = -1$) is presently no longer allowed.
\end{enumerate}

% ================================================================================================================================ %
\subsection{Structure Input} \label{StrInp}

The model of the accelerator is put together by constructing a sequence of blocks of linear elements, non-linear elements, observation points, and possibly a cavity with the keyword \texttt{CAV} used if this name does not appear in the list of single elements (\ref{SinEle}) with type $\pm 12$.\index{structure input}\index{STRU}
In that case, its parameters are given in the \textit{Synchrotron Oscillations} input block (\ref{SynOsc}).

\paragraph{Format} \texttt{\{ structure-element $\vert$ CAV $\vert$ GO \}}\index{CAV}\index{GO}

\bigskip
\begin{longtabu}{@{}lp{0.70\linewidth}}
    \texttt{structure-element} & Structure elements must appear as non-linear and observation elements in the single element list or in the list of blocks of the \textit{Block Definition} input block (\ref{BloDef}). \\
    \texttt{CAV} & A cavity can be introduced by a keyword \texttt{CAV}. This element does not appear in the single element list   (\ref{SinEle}). \\
    \texttt{GO} & Starting point: the keyword \texttt{GO} denotes where the tracking is started and where the tracked coordinates are recorded at each turn.
\end{longtabu}

\paragraph{Remarks}
Repetition of parts of the structure is indicated by parentheses with a multiplying factor N in front of them.
If the left parenthesis ``('' occurs in a line of input, the factor \texttt{N} is expected to be found in the preceding characters.
If the characters are blank, \texttt{N} is set to 1.\index{bracket expansion}
The right parenthesis ``)'' signals the end of the sequence to be repeated.

% ================================================================================================================================ %
\subsection{Displacement of Elements} \label{DisEle}

This block allows to displace nonlinear elements in horizontal and vertical positions.\index{displacement of elements}\index{DISP}
With the r.m.s. values of the horizontal and vertical displacements it is possible to achieve a displacement that is different from element to element.

To simulate a measured closed orbit at the position of non-linear elements, it is convenient to use the \textit{Displacement of Elements} input block instead of trying to produce a closed orbit by dipole kicks.\index{closed orbit}

\bigskip
\begin{tabular}{@{}lp{0.80\linewidth}}
    \textbf{Keyword} & \texttt{DISP} \\
    \textbf{Data lines} & Variable
\end{tabular}

\paragraph{Format} \texttt{name xd xdrms yd ydrms}

\bigskip
\begin{tabular}{@{}lp{0.80\linewidth}}
    \texttt{name}  & Name of the element which is displaced. \\
    \texttt{xd}    & Horizontal displacement [mm]. \\
    \texttt{xdrms} & r.m.s. of horizontal displacement [mm]. \\
    \texttt{yd}    & Vertical displacement [mm]. \\
    \texttt{ydrms} & r.m.s. of vertical displacement [mm].
\end{tabular}

\bigskip
\noindent In the case of an AC dipole\index{AC dipole} these variables are not meant for displacing this element but are used for the following AC dipole parameters:

\paragraph{Format} \texttt{name nfree nramp1 nplato nramp2}

\bigskip
\begin{tabular}{@{}lp{0.80\linewidth}}
    \texttt{name}   & May contain up to 48 characters. \\
    \texttt{nfree}  & Number of turns free of excitation at the beginning of the run. \\
    \texttt{nramp1} & Number of turns to ramp up the excitation amplitude from 0 to \texttt{ACdipAmp}. \\
    \texttt{nplato} & Number of turns of constant excitation amplitude. \\
    \texttt{nramp2} & Number of turns to ramp down the excitation amplitude.
\end{tabular}

\paragraph{Remarks}
In RACETRACK\index{RACETRACK} the displacements had been included in the \textit{Single Element} input block (\ref{SinEle}).
In SixTrack they must be given in the separate \textit{Displacement of Elements} input block because of the limited length of one line of input.

