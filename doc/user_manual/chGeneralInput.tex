
\chapter{General Input} \label{GenInp}

% ================================================================================================================================ %
\section{Program Version} \label{ProVer}

The \textit{Program Version} input block determines if all of the input will be in the input file \texttt{fort.3}, or if the geometry part of the machine (see~\ref{MaGe}) will be in a separate file: \texttt{fort.2}.\index{input block}\index{fort.2}\index{fort.3}
The latter option is useful if tracking parameters are changed, but the geometry part of the input is left as it is.
The geometry part can be produced directly from a MAD-X input file (see~\ref{MADT}).
Note that this line should not have a \texttt{NEXT} keyword after it.\index{NEXT}

\bigskip
\begin{tabular}{@{}lp{0.8\linewidth}}
    \textbf{Keyword}    & \texttt{FREE} or \texttt{GEOM} \index{FREE}\index{GEOM}\\
    \textbf{Data lines} & None \\
    \textbf{Format}     & \texttt{keyword comment title} \\
\end{tabular}

\paragraph{Format Description}~

\bigskip
\begin{tabular}{@{}lp{0.8\linewidth}}
    \texttt{keyword} & The first four characters of the first line of the \texttt{fort.3} input file are reserved for the keyword. \texttt{FREE} for free format input with all input in \texttt{fort.3}, and \texttt{GEOM} if the geometry part is in file  \texttt{fort.2}. \\
    \texttt{comment} & Following the first four characters, 8 characters are reserved for comments \\
    \texttt{title}   & The next 60 characters are interpreted as the title printed at the top of the output file \texttt{fort.6}.
\end{tabular}

% ================================================================================================================================ %
\section{Print Selection} \label{PriSel}

The \texttt{PRIN} flag is deprecated, and replaced by the \texttt{PRINT} flag in the \texttt{SETTINGS} block.\index{PRIN}

% ================================================================================================================================ %
\section{Settings} \label{STSett}

The \textit{Print Selection} input block available in earlier version has been replaced with the \texttt{SETTINGS} block.\index{SETT}
This was done to allow for more options related to what output SixTrack produces in \texttt{fort.6}.\index{fort.6}
The \texttt{PRIN} flag is available as one of several options in this block.\index{PRIN}
However, for backwards compatibility, the \texttt{PRIN} flag is still accepted.

\bigskip
\begin{tabular}{@{}lp{0.8\linewidth}}
    \textbf{Keyword}    & \texttt{SETT} \\
    \textbf{Data lines} & Variable
\end{tabular}

\paragraph{Format Description}~

\bigskip
\begin{tabular}{@{}lp{0.8\linewidth}}
    \texttt{PRINT} & This causes the printing of the input data to the output file \texttt{fort.6}. \\
    \texttt{DEBUG} & A global debug flag that causes the majority of the blocks to echo back the value read from the input file after parsing. It may also print out secondary values set based on input values read, or modification made to input values based on other dependencies and criteria.\index{DEBUG}\\
    \texttt{QUIET} & Followed by an integer specifying how ``quiet'' the output should be. A higher value causes less information to be printed back out. If the keyword is not present, the default value is 0, which means it is disabled. If it is present, but the integer value is omitted, it is set to be 1. This flag does not interfere with the \texttt{PRINT} flag.\index{QUIET}\\
    \texttt{PRINT\_DCUM} & This will cause the calculated s-coordinate of each structure element to be printed to \texttt{fort.6}. \\
    \texttt{PARTSUMMARY} & Enable or disable the printing of a particle summary after tracking. The flag takes an optional parameter to explicitly state whether it is ON or OFF. If omitted, it is assumed the user requests it to be ON. If the flag is omitted entirely, the default behaviour is determined by the particle count. If SixTrack is running with 64 particles or less, it is ON by default, otherwise OFF.\\
    \texttt{INITIALSTATE} & Followed by either ``binary'' or ``text''. This will write a file before tracking containing the initial state of all particles to either a binary or a text file.\\
    \texttt{FINALSTATE} & Followed by either ``binary'' or ``text''. This will write a file after tracking containing the final state of all particles, including those lost during tracking, to either a binary or a text file.\\
\end{tabular}


% ================================================================================================================================ %
\section{Comment Line} \label{ComLin}

An additional comment can be specified with the \textit{Comment} block.
The comment will be written to the binary data files (Appendix~\ref{Header}), and will appear in the post processing output as well.

\bigskip
\begin{tabular}{@{}lp{0.8\linewidth}}
    \textbf{Keyword}    & \texttt{COMM}\index{COMM}\index{comment} \\
    \textbf{Data lines} & 1 \\
    \textbf{Format}     & A string of up to 80 characters.
\end{tabular}

% ================================================================================================================================ %
\section{Iteration Errors} \label{IteErr}

For the processing procedures, the number of iterations and the precision to which the processing is to be performed are chosen with the \textit{Iteration Errors} input block.\index{iteration errors}
If the input block is left out, default values will be used.

\bigskip
\begin{tabular}{@{}lp{0.8\linewidth}}
    \textbf{Keyword}    & \texttt{ITER}\index{ITER} \\
    \textbf{Data lines} & 1 to 4 \\
    \textbf{Format}     & Each data line holds three values as in table~\ref{T-IteErr}, except for the fourth line where the horizontal and vertical aperture limits can be additionally specified. This has been added to avoid artificial crashes for special machines.
\end{tabular}

\begin{table}[h]
    \caption{Iteration Errors\index{chromaticity}\index{dispersion}\index{aperture limit}\index{closed orbit}}
    \label{T-IteErr}
    % \footnotesize
    \centering
    \renewcommand{\arraystretch}{1.5}
    \begin{tabular}{|l|l|l|>{\raggedright\arraybackslash}p{10.2cm}|}
        \hline
        \rowcolor{blue!30}
        \textbf{Variable} & \textbf{Type} & \textbf{Default} & \textbf{Description} \\
        \rowcolor{gray!15}
        \multicolumn{4}{|l|}{Data Line 1} \\
        \hline
        \texttt{ITCO} & \texttt{int} & \texttt{50}    & Number of Iterations for closed orbit calculation. \\
        \hline
        \texttt{DMA}  & \texttt{dbl} & \texttt{1e-12} & Demanded Precision of closed orbit displacements. \\
        \hline
        \texttt{DMAP} & \texttt{dbl} & \texttt{1e-15} & Demanded Precision of derivative of closed orbit displacements. \\
        \hline
        \rowcolor{gray!15}
        \multicolumn{4}{|l|}{Data Line 2} \\
        \hline
        \texttt{ITQV} & \texttt{int} & \texttt{10}    & Number of Iterations for Q Adjustment. \\
        \hline
        \texttt{DKQ}  & \texttt{dbl} & \texttt{1e-10} & Variations of quadrupole strengths. \\
        \hline
        \texttt{DQQ}  & \texttt{dbl} & \texttt{1e-10} & Demanded Precision of tunes. \\
        \hline
        \rowcolor{gray!15}
        \multicolumn{4}{|l|}{Data Line 3} \\
        \hline
        \texttt{ITCRO} & \texttt{int} & \texttt{10}    & Number of Iterations for chromaticity correction. \\
        \hline
        \texttt{DSM0}  & \texttt{dbl} & \texttt{1e-10} & Variations of sextupole strengths. \\
        \hline
        \texttt{DECH}  & \texttt{dbl} & \texttt{1e-10} & Demanded Precision of chromaticity correction. \\
        \hline
        \rowcolor{gray!15}
        \multicolumn{4}{|l|}{Data Line 4} \\
        \hline
        \texttt{DE0} & \texttt{dbl} & \texttt{1e-9} & Variations of momentum spread for chromaticity calculation. \\
        \hline
        \texttt{DED} & \texttt{dbl} & \texttt{1e-9} & Variations of momentum spread for evaluation of dispersion. \\
        \hline
        \texttt{DSI} & \texttt{dbl} & \texttt{1e-9} & Demanded Precision of desired orbit r.m.s. value; compensation of resonance width. \\
        \hline
        \texttt{APER(1)} & \texttt{dbl} & \texttt{1000 [mm]} & Demanded Precision of horizontal aperture limit. \\
        \hline
        \texttt{APER(2)} & \texttt{dbl} & \texttt{1000 [mm]} & Demanded Precision of vertical aperture limit. \\
        \hline
    \end{tabular}
    % \normalsize
\end{table}

% ================================================================================================================================ %
\section{MAD-X to SixTrack Conversion} \label{MADT}

A converter has been developed~\cite{CONVERTOR}, which is directly linked to MAD-X\@.\index{MAD-X}
It produces the geometry file \texttt{fort.2}; an appendix to the parameter file \texttt{fort.3}, which defines which of the multipole errors are switched on; the error file \texttt{fort.16}, and the file \texttt{fort.8} which holds the transverse misalignments and the tilt of the non-linear kick elements.\index{fort.8}\index{fort.16}\index{multipole errors}
It also produce a file \texttt{fort.34} with linear lattice functions, phase advances and multipole strengths needed for resonance calculations for the program \textit{SODD}~\cite{SODD}.\index{fort.34}\index{resonance calculations}

% A description of %how to use the converter can be found in Appendix\ref{MADSIXC}.
% Pasted here for reference:

% \chapter{MAD SixTrack Convertor} \label{MADSIXC}

% A special version of the MAD program has been generated that creates
% input files for SixTrack with the complete and exact description of
% the current accelerator model stored in MAD\@. This then allows a
% detailed comparison of the tracking results from both programs.

% The programs SixTrack and MAD \cite {MAD} are both used for particle
% tracking in realistic machines such as the LHC with all expected field
% errors up to the highest order, and with displacement errors of
% quadrupoles and possibly other elements. The LHC lattice and error
% description, however, exists only in MAD format due to its much
% greater flexibility and readability. If one wants to compare tracking
% results of the two programs one needs therefore a conversion of the
% current machine inside MAD into SixTrack format. In particular the
% generated random field and displacement errors have to be transmitted
% correctly since they influence the outcome of the long-term tracking
% greatly.

% To this end it was necessary to make a special MAD version that reads
% a machine description, expands it, generates the errors, and then
% performs the necessary transformations for SixTrack: different
% conventions for the signs of magnet forces, grouping of identical
% elements, concatenation of multiple drift spaces, creation of blocks
% of linear elements between nonlinear ones, a different description
% of field and position errors, and similar tedious questions had to be
% solved. The results of the transformation are then written into input
% files for SixTrack.

% The special MAD version is actually a full MAD (except for plotting)
% with an additional module for the SixTrack output.  This then allows
% the user to perform orbit corrections and other calculations inside
% MAD, and have the resulting modifications of the machine be
% transferred to SixTrack.  All position errors (including tilt errors)
% and field errors are transmitted.

% The converter works correctly for thin and thick lens lattices.  In
% the latter case, all dipoles and quadrupoles are split into two, and a
% multipole is inserted in the centre, except for orbit corrector
% magnets which are converted to thin lens.  Since MAD does not support
% field errors higher than third order for thick elements, the
% multipolar errors will have to be created with the corresponding thin
% lens model, or by some other means.

% The special MAD version accepts one additional command:
% \begin{verbatim}
%  sixt:
%       elem       = (up to 20) element names
%       lgelem     = (up to 20) lengths of the above elements
%       structure  = structure file name, default: "fort.2"
%       param_in   = name of a (mandatory, possibly empty)
%                    extra input file for MAD, default: "fort.3.in"
%       param_out  = parameter file name, default: "fort.3"
%       errors     = field errors file name, default: "fort.16"
%       poserr     = position errors file name, default: "fort.8"
%       cavall     = all cavities in the structure, default: one only.
% \end{verbatim}
% The element length facility is necessary to give SixTrack the real
% lengths of the dipoles (and possibly other elements) that are replaced
% by thin multipoles.  \\
% The parameter input file is copied to the start of the parameter
% output file which may be useful in certain situations.
