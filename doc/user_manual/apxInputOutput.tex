\chapter{Input and Output Files} \label{Files}

The program uses a couple of files for its input and output procedures.

\begin{center}
\renewcommand{\arraystretch}{1.5}
\begin{longtable}{|c|c|c|c|>{\raggedright\arraybackslash}p{7.8cm}|}
    \caption{List of Input and Output Files.}\\
    \hline

    \rowcolor{blue!30}
    \textbf{File Unit} & \textbf{Input} & \textbf{Output} & \textbf{File Type} & \textbf{Contents} \\
    \hline
    \endfirsthead

    \hline
    \rowcolor{blue!30}
    \textbf{File Unit} & \textbf{Input} & \textbf{Output} & \textbf{File Type} & \textbf{Contents} \\
    \hline
    \endhead

    2 & \checkmark & & Ascii & Geometry and Strength Parameters \\
    \hline
    3 & \checkmark & & Ascii & Tracking Parameters \\
    \hline
    4 & & \checkmark & Ascii & Geometry and strength Parameters (format as file \texttt{fort.2}) \\
    \hline
    6 & & \checkmark & Ascii & Standard Output \\
    \hline
    8 & \checkmark & & Ascii & Name, hor., ver. Misalignment and Tilt \\
    \hline
    9 & & \checkmark & Ascii & Internally used multipoles Format: $a16,\ 2 \times \{6 \times (1p,3d23.15), (1p,2d23.15)\}$\\
    \hline
    10 & \checkmark & \checkmark & Ascii & Summary of Post-processing (auxiliary) \\
    \hline
    % 11 & & \checkmark & Ascii & This file is used to dump linear coupling parameters at locations of choice \\
    % \hline
    12 & & \checkmark & Ascii & End coordinates of all particles. Format: ($15 \times F10.6$) \\
    \hline
    13 & \checkmark & & Ascii & Start coordinates for a prolongation \\
    \hline
    14 & & \checkmark & Ascii & Horizontal FFT spectrum for detailed analysis. Format: ($2 \times F10.6$) \\
    \hline
    15 & & \checkmark & Ascii & Vertical FFT spectrum for detailed analysis. Format: ($2 \times F10.6$) \\
    \hline
    16 & \checkmark & & Ascii & External multipole errors. Format: $a16,\ 2 \times \{6 \times (1p,3d23.15),(1p,2d23.15)\}$ \\
    \hline
    17 & & \checkmark & Ascii & Additional Map at location of interest \\
    \hline
    18 & & \checkmark & Ascii & One turn map with differential algebra \\
    \hline
    19 & \checkmark & \checkmark & Ascii & Internal use for Differential Algebra \\
    \hline
    20 & & \checkmark & Meta-file & PS-file of selected Plots \\
    \hline
    21 & & \checkmark & Ascii & Factorisation of the one turn map \\
    \hline
    22 & & \checkmark & Ascii & Transformation in the Normal Form coordinate \\
    \hline
    23 & & \checkmark & Ascii & Hamiltonian in action variables \\
    \hline
    24 & & \checkmark & Ascii & Tune-shift in action coordinates \\
    \hline
    25 & & \checkmark & Ascii & Tune-shift in Cartesian coordinates \\
    \hline
    26 & & \checkmark & Binary & Binary version of unit 18 \\
    \hline
    27 & & \checkmark & Ascii & Name, hor., ver. misalignment and tilt \\
    \hline
    28 & & \checkmark & Ascii & Horizontal closed orbit displacement, measured at monitors \\
    \hline
    29 & & \checkmark & Ascii & Vertical closed orbit displacement, measured at monitors \\
    \hline
    30 & \checkmark & & Ascii & Name, random strength, misalignments and tilt \\
    \hline
    31 & & \checkmark & Ascii & Name, random strength, misalignments and tilt \\
    % \hline
    % 32 & \checkmark & \checkmark & Binary & Binary dump of full accelerator description \\
    \hline
    33 & \checkmark & & Ascii & Guess values for 6D closed orbit search \\
    \hline
    % 34 & & \checkmark & Ascii & Multipole strength and linear lattice parameters~\cite{SODD} \\
    % \hline
    90--k & & \checkmark & Binary & Tracking Data (not singletrackfile) $0 \leq k \leq 31$ \\
    \hline
    90 & & \checkmark & Binary & Tracking Data (singletrackfile) \texttt{singletrackfile.dat} \\
    \hline
    % 92 & & \checkmark & Ascii & Checkpoint/Restart only: Program ``standard output'' (lout) \\
    % \hline
    % 93 & & \checkmark & Ascii & Checkpoint/Restart only: log file \\
    % \hline
    % 94 & & \checkmark & Ascii & Checkpoint/Restart only: Temp file for resetting binary tracking data file(s) \\
    % \hline
    % 95 & \checkmark & \checkmark & Binary & Checkpoint/restart only: data file 1 \\
    % \hline
    % 96 & \checkmark & \checkmark & Binary & Checkpoint/restart only: data file 2 \\
    % \hline
    % 98 & & \checkmark & Ascii & 6D coordinates at Cavity (\texttt{1p,6(2x,e25.18)}) \\
    % \hline
\end{longtable}
\end{center}

In addition to those files listed in the table, DUMP uses arbitrary file unit numbers as determined by the input file.
The collimation module also uses many input/output files at various units, which are not listed here.
