
\chapter{External Tools} \label{ExtTools}

SixTrack supports interfacing with external libraries and simulation tools.
This chapter describes how to use these tools with SixTrack.

\section{Pythia Integration} \label{ExtPythia}

\textcolor{notered}{\textbf{Note:} This module should be considered experimental. It is not yet fully tested for correct physics.}

Pythia8 is ``a standard tool for the generation of high-energy collisions, comprising a coherent set of physics models for the evolution from a few-body hard process to a complex multihadronic final state''~\cite{pythia8}\index{Pythia}.
Version 8 is rewritten in c++ as a library, which makes it suitable for direct integration into SixTrack 5 via the update and extended c-interface available in Fortran2008.
The integration was made specifically for the Scatter module\index{scatter} to function as an event generator for single diffractive scattering.

The Pythia8 library is built using the \texttt{buildLibraries.sh} script, and linking with SixTrack is provided with the \texttt{PYTHIA} compiler flag\index{build}.
When the \texttt{PYTHIA} block is present in \texttt{fort.3}, a new generator becomes available to the Scatter module.

Note that the proton mass used by PYTHIA may not be the same as the proton mass in SixTrack.
This may cause differences in the energy to momentum conversion.
The mass from PYTHIA is printed to stdout during execution, and can also be found in the \texttt{scatter\_summary.dat} file.

\paragraph{Particle Species}~ \texttt{SPECIES beam1 beam2}\\

Set the particle species for Beam 1 and Beam 2 (or the target, if not a beam).
These are the species used for the PYTHIA event generator, and in principle can be anything that PYTHIA supports, even if SixTrack cannot track it.

Supported particle species are: \texttt{PROTON}, \texttt{ANTIPROTON}, \texttt{NEUTRON}, \texttt{ANTINEUTRON}, \texttt{PION+}, \texttt{PION-}, \texttt{PION0}, \texttt{PHOTON}, \texttt{ELECTRON}, \texttt{POSITRON}, \texttt{MUON-}, and \texttt{MUON+}.

\paragraph{Beam Energy}~ \texttt{ENERGY beam1 [beam2]}\\

The energy of the beams in MeV.
If only Beam 1 is specified, the value is taken as the centre of mass energy by PYTHIA.

\paragraph{Use Tracked Particles}~ \texttt{REALBEAM on|off}\\

By default, PYTHIA generates events for head on collisions at reference energy.
The event is then projected back onto the tracked particle in SixTrack.
If, instead, you want to send the actual tracked particle to SixTrack, together with a sample particle generated from a beam distribution, set the \texttt{REALBEAM} flag to ``on''.

\paragraph{Process Selection}~ \texttt{PROCESS type [crossSection]}\\

The scattering process for the event generator can be set with this keyword.
One process for each line.
If the \texttt{MODES} are set to \texttt{MANUAL}, the cross sections can be specified as a second argument.

\bigskip
\noindent Available process types are:
\begin{itemize}
  \item Elastic: \texttt{EL}, \texttt{ELASTIC}
  \item Single diffractive: \texttt{SD}, \texttt{SINGLEDIFF},\texttt{SINGLEDIFFRACTIVE}
  \item Double diffractive: \texttt{DD}, \texttt{DOUBLEDIFF},\texttt{DOUBLEDIFFRACTIVE}
  \item Central diffractive: \texttt{CD}, \texttt{CENTRALDIFF},\texttt{CENTRALDIFFRACTIVE}
  \item Non-diffractive: \texttt{ND}, \texttt{NONDIFF},\texttt{NONDIFFRACTIVE}
\end{itemize}

\paragraph{Coulomb}~ \texttt{COULOMB on|off [tmin]}\\

Enable or disable the Coulomb part of the elastic scattering generator.
A \texttt{tmin} is required, but if it is not provided here, it defaults to $5\times 10^{-5}$.

\paragraph{Total Cross Section}~ \texttt{SIGTOTAL crossSection}\\

The total cross section can be provided in units of millibarn.
This is only relevant when the \texttt{MODES} are set to \texttt{MANUAL}.
The total cross section is then sent to PYTHIA with the cross sections provided with the \texttt{PROCESS} settings.

\paragraph{Total Cross Section Mode}~ \texttt{SIGTOTALMODE mode}\\

Set the total cross section mode for PYTHIA.
The options are: \texttt{MANUAL}, \texttt{DL}, \texttt{MBR}, \texttt{ABMST}, and \texttt{RPP2016}.
Alternatively, an integer value can be provided.
For a description of these options, and their integer equivalents, please consult the PYTHIA manual~\cite{pythia8}.

\paragraph{Diffractive Cross Section Mode}~ \texttt{SIGDIFFMODE mode}\\

Set the diffractive cross section mode for PYTHIA.
The options are: \texttt{MANUAL}, \texttt{SAS}, \texttt{MBR}, and \texttt{ABMST}.
Alternatively, an integer value can be provided.
For a description of these options, and their integer equivalents, please consult the PYTHIA manual~\cite{pythia8}.

\paragraph{Random Seed}~ \texttt{SEED value}\\

The random number seed to send to the PYTHIA library.
This has no relation to the internal SixTrack random seeds.

\paragraph{Settings File}~ \texttt{FILE fileName}\\

If the \texttt{PYTHIA} block does not provide sufficient settings, a PYTHIA settings file can be set instead using this command.

\paragraph{Example}~\\
\begin{cverbatim}
PYTHIA------------------------------------------------------------------
  SPECIES PROTON PROTON
  ENERGY 7000000 7000000
  SEED 41
  PROCESS ELASTIC
  PROCESS SINGLEDIFFRACTIVE
NEXT

SCATTER-----------------------------------------------------------------
  GEN  sc_elastic  PPBEAMELASTIC 0.046  18.52  4.601  2.647  0.2  8.0
  GEN  sc_pythia   PYTHIA  27.0
  PRO  ipFLAT      FLAT   3e25  938.0  7000000
  ELEM ip1_scatter ipFLAT auto 1.0 sc_elastic
  ELEM ip5_scatter ipFLAT auto 1.0 sc_pythia
NEXT
\end{cverbatim}
