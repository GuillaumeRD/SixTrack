\chapter{Introduction}

The Single Particle Tracking Code SixTrack is optimised to carry two particles~\footnote{%
  Two particles are needed for the detection of  chaotic behaviour.
} through an accelerator structure over a large number of turns.
It is an offspring of RACETRACK \cite{RACETRACK} written by Albin Wrulich.
The input structure has been changed as little as possible so that slightly modified RACETRACK input files, or those of other offsprings like FASTRAC \cite{FASTRAC} can be read.

The main features of SixTrack are: \bigskip
\begin{enumerate}
  \item Treatment of the full six--dimensional motion including synchrotron motion in a symplectic manner \cite{Ripken85}.
        The energy can be ramped at the same time considering the relativistic change of the velocity \cite{Ripken87}.
  \item Detection of the onset of chaotic motion and thereby the long term dynamic aperture by evaluating the Lyapunov exponent.
  \item Post processing procedure allowing:
  \begin{itemize}
    \item calculation of the Lyapunov exponent,
    \item calculation of the average phase advance per turn,
    \item FFT analysis,
    \item resonance analysis,
    \item calculation of the average, maximum and minimum values of the Courant--Snyder emittance and the invariants of linearly coupled motion,
    \item calculation of smear, and
    \item plotting using the CERN packages HBOOK, HPLOT and HIGZ \cite{HBOOK,HPLOT,HIGZ}
  \end{itemize}
  \item Calculation of first--order resonances and of correction schemes for the resonances \cite{Gilbert78}.
  \item Calculation of the one--turn map using the differential algebra techniques.
        The original DA package by M.Berz \cite{Berz89} has been replaced by the package of LBL~\cite{DALIE}.
        The Fortran code is transferred into a Map producing via the (slightly modified) ``DAFOR'' code~\cite{DAFOR}.
  \item The code is vectorised, with two particles, the number of amplitudes, the different relative momentum deviations \mbox{$ \frac{\Delta p}{p_o} $} in parallel \cite{Sixvec}.
  \item Operational improvements:
  \begin{itemize}
    \item free format input,
    \item optimisation of the calculation of multipole kicks,
    \item improved treatment of random errors,
    \item each binary data file has a header describing the history of the run (Appendix~\ref{Header})
  \end{itemize}
\end{enumerate}

The SixTrack input is line oriented.
Each line of 80 characters is treated as one string of input in which a certain sequence of numbers and character strings is expected to be found.
The numbers and character strings must be separated by at least one blank.
Floating point numbers can be given in any format, but must be distinguished from integer numbers.
Omitted values at the end of an input line will keep their default values (~\ref{DTP}).
Lines with a slash ``/'' in the first column will be ignored by the program.

For detailed questions concerning rounding errors, calculation of the Lyapunov exponent and determination of the long term dynamic aperture, see \cite{thesis}.

In chapter~\ref{InpStr}, the input structure of SixTrack is discussed in detail.
To facilitate the use of the program, a set of appendices are added,
  giving a list of keywords \mbox{(Appendix~\ref{Lkey}),}
  a list of default values \mbox{(Appendix~\ref{Default}),}
  the input and output files \mbox{(Appendix~\ref{Files}),}
  a description of the data structure of the binary data files \mbox{(Appendix~\ref{Header}),}
  and tracking examples \mbox{(Appendix~\ref{Exam}).}
