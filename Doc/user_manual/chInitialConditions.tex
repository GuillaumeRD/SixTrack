
\chapter{Initial Conditions for Tracking}

\subparagraph{Description} For the study of nonlinear system the
choice of initial conditions is of crucial importance. The input
structure for the initial conditions was therefore organise in such a
way as to allow for maximum flexibility. SixTrack is optimised to
reach the largest possible number of turns. In order to derive the
Lyapunov exponent and thereby to distinguish between regular and
chaotic motion, the particle has a close by companion particle.
Moreover, experience has shown that varying only the amplitude while
keeping the phases constant is sufficient to understand the nonlinear
dynamics, as a subsequent detailed post--processing allows to find the
dependence of the parameter of interest on these phases.

\section{Tracking Parameters} \label{TraPar}

\subparagraph{Description} All tracking parameters are defined with
this input block, the initial coordinates are generally set here, too.
A fine tuning of the initial condition is done with Initial
Coordinates block (~\ref{IniCoo}) and the parameters for the
synchrotron oscillation are given in block (~\ref{SynOsc})

\subparagraph{Keyword} TRAC \subparagraph{Number of data lines} 3

\subparagraph{Format}
\begin{itemize}
\item data line 1: {\em numl numlr napx amp(1) amp0 ird imc niu(1) niu(2) numlcp numlmax}
\item data line 2: {\em idy(1) idy(2) idfor irew iclo6} \/(integers)
\item data line 3: {\em nde(1) nde(2) nwr(1) nwr(2) nwr(3) nwr(4) ntwin ibidu iexact} \/(integers)
\end{itemize}

\begin{description}
\item [numl] (integer) Number of turns in the forward direction
\item [numlr] (integer) Number of turns in the backward direction
\item [napx] (integer) Number of amplitude variations (i.e.\ particle pairs)
\item [amp(1), amp0] (floats) Start and end amplitude (any sign) in
  the horizontal phase space plane for the amplitude variations. The
  vertical amplitude is calculated using the ratio between the
  horizontal and vertical emittance set in the {\em Initial
    Coordinates} \/block (~\ref{IniCoo}), where the initial phase in
  phase space are also set. Additional information can be found in the
  {\em Remarks}\/.
\item [ird] (integer) Ignored.
\item [imc] (integer) Number of variations of the relative momentum
  deviation \mbox{$ \frac{\Delta p}{p_o} $}.  The maximum value of the
  relative momentum deviation \mbox{$ \frac{\Delta p}{p_o} $} is taken
  from that of the first particle in the {\em Initial Coordinates}
  \/block (~\ref{IniCoo}).  The variation will be between \mbox{$ \pm
    \frac{\Delta p}{p_o} (\mathrm{max}) $} in steps of \mbox{$
    \frac{\Delta p}{p_o} (\mathrm{max}) $ / ({\em imc--1}\/).}
\item[niu(1), niu(2)] Unknown; default values are 0.
\item[numlcp] Checkpoint/restart version: How often to write checkpointing files.
\item[numlmax] Checkpoint/restart version: Maximum ammount of turns; default is $10^6$.
\item [idy(1), idy(2)] A tracking where one of the transversal motion
  planes shall be ignored is only possible when all coupling terms are
  switched off.  The part of the coupling that is due to closed orbit
  and other effects can be turned off with these switches.
 \begin{itemize}
 \item {\em idy(1), idy(2)} \/= 1 : coupling on
 \item {\em idy(1), idy(2)} \/= 0 : coupling to the horizontal and
   vertical motion plane respectively switched off
 \end{itemize}
\item [idfor] Usually the closed orbit is added to the initial
  coordinates. This can be turned off using {\em idfor}\/, for
  instance when a run is to be prolonged.
 \begin{itemize}
 \item {\em idfor} \/= 0 : closed orbit added.
 \item {\em idfor} \/= 1 : initial coordinates unchanged.
 \item {\em idfor} \/= 2 : prolongation of a run, taken the initial coordinates from unit \# 13.
 \end{itemize}
\item [irew] To reduce the amount of tracking data after each
  amplitude and relative momentum deviation iteration \mbox{$
    \frac{\Delta p}{p_o} $} the binary output units 90 and lower (see
  Appendix~\ref{Files}) are rewound.  This is always done when the
  post--processing is activated (~\ref{PosPro}). For certain
  applications it may be useful to store all data. The switch {\em
    irew} \/allows for that.
 \begin{itemize}
 \item {\em irew} \/= 0 : unit 90 (and lower) rewound
 \item {\em irew} \/= 1 : all data on unit 90 (and lower)
 \end{itemize}
\item [iclo6]
  This switch allows to calculate the 6D closed orbit and optical functions at the starting point, using the differential algebra package.
  % It is ignored in the regular tracking versions.
  It is active in all versions that link to the Differential Algebra package.
  % This 6D closed orbit can be calculated from any longitudinal position contrary to earlier versions.
  Note that iclo6 > 0 is mandatory for 6D simulations, and that iclo6 == 0 is mandatory for 4D simulations.
 \begin{itemize}
 \item {\em iclo6} \/= 0 : switched off
 \item {\em iclo6} \/= 1 : calculated
 \item {\em iclo6} \/= 2 : calculated and added to the initial coordinates (~\ref{IniCoo}).
 \item {\em iclo6} \/= 5 or = 6: like for {\em 1} \/and {\em 2} \/ but in addition a guess closed orbit is read (in free format) from fileunit \# 33.
 \end{itemize}
\item [nde(1)]
  Number of turns at flat bottom, useful for energy ramping.
\item [nde(2)]
  Number of turns for the energy ramping.  {\em numl}\/--{\em nde(2)} \/ gives the number of turns on the flat top.
  For constant energy with \mbox{$ nde(1) = nde(2) = 0 $} the particles are considered to be on the flat top.
\item [nwr(1)]
  Every {\em nwr(1)}\/'th turn the coordinates will be written on unit 90 (and lower) in the flat bottom part of the tracking.
\item [nwr(2)]
  Every {\em nwr(2)}\/'th turn the coordinates in the ramping region will be written on unit 90 (and lower).
\item [nwr(3)]
  Every {\em nwr(3)}\/'th turn at the flat top a write out of the coordinates on unit 90 (and lower) will occur.
  For
  constant energy this number controls the amount of data on unit 90
  (and lower), as the particles are considered on the flat top.
\item [nwr(4)] In cases of very long runs it is sometimes useful to
  save all coordinates for a prolongation of a run after a possible
  crash of the computer.  Every {\em nwr(4)}\/'th turn the coordinates
  are written to unit 6.
\item [ntwin] For the analysis of the Lyapunov exponent it is usually
  sufficient to store the calculated distance of phase space together
  with the coordinate of the first particle ({\em ntwin} \/set to
  one). You may want to improve the 6D calculation of the distance in
  phase space with {\em sigcor, dpscor} \/(see~\ref{IniCoo}) when the
  6D closed orbit is not calculated with {\em iclo6} \/$\neq 2$. If
  storage space is no problem, one can store the coordinates of both
  particles ({\em ntwin} \/set to two). The distance in phase space is
  then calculated in the post--processing procedure
  (see~\ref{PosPro}). This also allows a subsequent refined Lyapunov
  analysis using differential--algebra and Lie--algebra techniques
  (\cite{Refine}).
\item [ibidu] Switch to creat or read binary dump of the full
  accelerator decription on file \# 32. The parameters relevant to
  tracking, i.e.\ {\em numl, amp0, amp(1), amp(2), damp, chi0, chid,
    rat, $x_1$, $x'_1$, $y_1$, $y'_1$, ${\sigma}_1$, $\frac{\Delta
      p}{p_{o1}}$, $x_2$, $x'_2$, $y_2$, $y'_2$, ${\sigma}_2$,
    $\frac{\Delta p}{p_{o2}}$, time0, time1}, are to be given via the
  tracking parameter file \# 3.
 \begin{itemize}
 \item {\em ibidu} \/= 1 : write dump
 \item {\em ibidu} \/= 2 : read dump
 \end{itemize}
\item [iexact] Switch to enable exact solution of the equation of motion into
  tracking and 6D (no 4D) optics calculations.
  \begin{itemize}
    \item {\em iexact} \/= 0 : approximated equation
              (e.g. $x'\simeq \frac{P_x}{P_0(1+\delta)}$,
                    $y'\simeq \frac{P_y}{P_0(1+\delta)}$);
    \item {\em iexact} \/= 1 : exact equation (e.g
      $x'\simeq \frac{P_x}{P_0\sqrt{(1+\delta)^2-P_x^2-P_y^2}}$,
      $y'\simeq \frac{P_y}{P_0\sqrt{(1+\delta)^2-P_x^2-P_y^2}}$ ).
 \end{itemize}
\end{description}

\subparagraph{Remarks}
\begin{enumerate}
\item This input data block is usually combined with the {\em Initial
    Coordinates} \/input block (~\ref{IniCoo}) to allow a flexible
  choice of the initial coordinates for the tracking.
\item For a prolongation of a run the following parameters have to be
  set :
\begin{itemize}
\item in this input block : idfor = 1
\item in the {\em Initial coordinates} \/input block :
\begin{enumerate}
\item itra = 0
\item take the end coordinates of the previous run as the initial
  coordinates (including all digits) for the new run.
\end{enumerate}
\end{itemize}
\item A feature is installed for a prolongation of a run by using
  \mbox{\em idfor = 2} \/and 
  reading the initial data from unit \# 13. The end coordinates are
  now written on unit \# 12 after each run. Intermediate coordinates
  are also written on unit \# 12 in case the turn number \mbox{\em
    nwr(4)} \/is exceeded in the run. The user takes responsibility to
  transfer the required data from unit \# 12 to unit \# 13 if a
  prolongation is requested.
\item Some illogical combinations of parameters have been suppressed.
\item The initial coordinates are calculated using a proper linear 6D
  transformation: {\em amp(1)} \/is still the maximum horizontal
  starting amplitude (excluding the dispersion contribution) from
  which the emittance of mode 1 $e_I$ is derived, {\em rat}
  \/(see~\ref{IniCoo}) is the ratio of $e_{II}/e_I$ of the emittances
  of the two modes. The momentum deviation $\frac{\Delta p}{p_{o1}}$ is used to
  define a longitudinal amplitude. The 6 normalized coordinates read:
\begin{itemize}
\item horizontal:\\

  $\sqrt{e_I}=\frac{amp(1)}
  {\sqrt{\beta_{xI}}+\sqrt{\left|rat\right|\times\beta_{xII}}}$,\\

  $0.$

\item vertical: \\

 $sign(rat)\times \sqrt{e_{II}}$, with $e_{II} =
 \left|rat\right|\times e_{I}$,\\

 $0.$

\item longitudinal: \\

$0.$,\\ 

$\frac{\Delta p}{p_{o1}} \times \sqrt{\beta_{sIII}}$

\end{itemize}
and are then transformed with the 6D linear transformation into real
space. Note that results may differ from those of older versions.

\item The amplitude scan is performed from \emph{amp(1)} to \emph{amp0} in steps of $\emph{delta} = (\emph{amp0} - \emph{amp(1)}) / (napx-1)$.
For the intermediate amplitudes, \emph{delta} is added up for each step, however the last amplitude is guaranteed to be fixed to the given value.
This enables ``control calculations'' by setting the first amplitude of one simulation equal to the last amplitude of another simulation, and unless there are calculation errors, they shall produce exactly the same results.

\item Note that if \emph{iclo6} = 2 and \emph{idfor} = 0 in the input file, then \emph{idfor} is internally set to 1, as is seen in some outputs.
This does not mean that the closed orbit is not added; the setting of \emph{iclo6} = 2 simply takes precedence.

\end{enumerate}

\section{Initial Coordinates} \label{IniCoo}

\subparagraph{Description} The {\em Initial Coordinates} \/input block
is meant to manipulate how the initial coordinates are organise, which
are generally set in the tracking parameter block (~\ref{TraPar}).
Number of particles, initial phase, ratio of the horizontal and
vertical emittances and increments of \mbox{2 $\times$ 6} coordinates
of the two particles, the reference energy and the starting energy for
the two particles.

\subparagraph{Keyword} INIT \subparagraph{Number of data lines} 16

\subparagraph{Format}
\begin{itemize}
\item first data line: {\em itra chi0 chid rat iver}
\item data lines 2 to 16: {\em 15 initial coordinates as listed in Table}~\ref{T-IniCoo}
\end{itemize}

\begin{description}
\item [itra] (integer) Number of particles
 \begin{itemize}
 \item {\em itra} \/= 0 : Amplitude values of tracking parameter block
   (~\ref{TraPar}) are ignored and coordinates of data line 2--16 are
   taken. {\em itra} \/is set internally to 2 for tracking with two
   particles.  This is necessary in case a run is to be prolonged.
 \item {\em itra} \/= 1 : Tracking of one particle, twin particle
   ignored
 \item {\em itra} \/= 2 : Tracking the two twin particles
 \end{itemize}
\item [chi0] Starting phase of the initial coordinate in the
  horizontal and vertical phase space projections
\item [chid] Phase difference between first and second particles
\item [rat] Denotes the emittance ratio ($e_{II}/e_I$) of horizontal
  and vertical motion. For further information see the \emph{Remarks} of the TRAC input block in Section~\ref{TraPar}.
\item [iver] In tracking with coupling it is sometimes desired to
  start with zero vertical amplitude which can be painful if the
  emittance ratio {\em rat} \/is used to achieve it. For this purpose
  the switch {\em iver} \/has been introduced:
\begin{itemize}
\item {\em iver} \/= 0 : Vertical coordinates unchanged
\item {\em iver} \/= 1 : Vertical coordinates set to zero.
\end{itemize}
\end{description}

\begin{table}
\caption{Initial Coordinates of the 2 Particles}
\label{T-IniCoo}
\centering
\begin{tabular}{|l|l|}
  \hline
  data line & contents \\
  \hline
  2 & $x_1$ [mm] coordinate of particle 1 \\
  3 & $x'_1$ [mrad] coordinate of particle 1 \\
  4 & $y_1$ [mm] coordinate of particle 1 \\
  5 & $y'_1$ [mrad] coordinate of particle 1 \\
  6 & path length difference 1 ($ {\sigma}_1 = s - v_o \times t$) [mm]
  of
  particle 1 \\
  7 & $ \frac{\Delta p}{p_{o1}} $ of particle 1 \\
  8 & $x_2$ [mm] coordinate of particle 2 \\
  9 & $x'_2$ [mrad] coordinate of particle 2 \\
  10 & $y_2$ [mm] coordinate of particle 2 \\
  11 & $y'_2$ [mrad] coordinate of particle 2 \\
  12 & path length difference ($ {\sigma}_2 = s - v_o \times t$) [mm]
  of
  particle 2 \\
  13 & $ \frac{\Delta p}{p_{o2}} $ of particle 2 \\
  14 & energy [MeV] of the reference particle \\
  15 & energy [MeV] of particle 1 \\
  16 & energy [MeV] of particle 2 \\
  \hline
\end{tabular}
\end{table}

\subparagraph{Remarks}
\begin{enumerate}
\item These 15 coordinates are taken as the initial coordinates if
  {\em itra} \/is set to zero (see above). If {\em itra} \/is 1 or 2
  these coordinates are added to the initial coordinates generally
  defined in the tracking parameter block (~\ref{TraPar}). This
  procedure seems complicated but it allows freely to define the
  initial difference between the two twin particles. It also allows in
  case a tracking run should be prolonged to continue with precisely
  the same coordinates. This is important as small difference may lead
  to largely different results.
\item The reference particle is the particle in the centre of the
  bucket which performs no synchrotron oscillations.
\item The energy of the first and second particles is given
  explicitly, again to make possible a continuation that leads
  precisely to the same results as if the run would not have been
  interrupted.
\item There is a refined way of prolonging a run, see the \mbox{\em
    Tracking Parameters} \/input block (~\ref{TraPar}).
\end{enumerate}

\section{Synchrotron Oscillation} \label{SynOsc}

\subparagraph{Description} The parameters needed for treating the
synchrotron oscillation in a symplectic manner are given in the {\em
  Synchrotron Oscillation} \/input block.

\subparagraph{Keyword} SYNC \subparagraph{Number of data lines} 2

\subparagraph{Format}
\begin{itemize}
\item first data line: {\em harm alc u0 phag tlen pma ition dppoff}
\item second data line: {\em dpscor sigcor}
\end{itemize}

\begin{description}
\item [harm] Harmonic number
\item [alc] Momentum compaction factor, used here only to calculate
  the linear synchrotron tune $ Q_{S} $.
\item [u0] Circumference voltage in [MV]
\item [phag] Acceleration phase in degrees
\item [tlen] Length of the accelerator in meters
\item [pma] rest mass of the particle in $ \mathrm{MeV}/\mathrm{c}^{2}
  $
\item [ition] (integer) Transition energy switch
 \begin{itemize}
 \item {\em ition} = \hspace{.3mm} 0 for no synchrotron oscillation
   (energy ramping still possible)
 \item {\em ition} = \hspace{.5mm} 1 for above transition energy
 \item {\em ition} = --1 for below transition energy
 \end{itemize}
\item [dppoff]  Offset Relative Momentum Deviation \mbox{$ \frac{\Delta
      p}{p_o} $}: a fixpoint with respect to synchrotron oscillations.
  It becomes active when the 6D closed orbit is calculated (see item
  {\it iclo6} \/in section~\ref{TraPar}).
\item [dpscor, sigcor] Scaling factor for relative momentum deviation
  \mbox{$ \frac{\Delta p}{p_o} $} and the path length difference
  ($\sigma = s - v_o \times t$) respectively.  They can be used to
  improve the calculation of the 6D distance in phase space, but is
  only used when {\em ntwin = 1} \/in the tracking parameter input
  block~(\ref{TraPar}). Please set to 1 when the 6D closed is
  calculated.
\end{description}
Note that the value of \emph{tlen} is also calculated internally by SixTrack (in ``dcum''), and a warning is issued if the given value is different from the calculated value.

