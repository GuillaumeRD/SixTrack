\chapter{List of Default Values} \label{Default}

% ================================================================================================================================ %
\section{Default Tracking Parameters} \label{DTP}

Some of the parameters for tracking are set to non-zero values.
This is done for instance to avoid as much as possible program errors such as division by zero due to an erroneous input.
The default values for the \textit{Iteration Errors} (\ref{IteErr}) see table~\ref{T-IteErr}.

\newcounter{dtp} \setcounter{dtp}{0}

\bigskip
\begin{table}[h]
    \caption{Default Tracking Parameters}
    \label{T-DTP}
    \scriptsize
    \centering
    \renewcommand{\arraystretch}{1.5}
    \begin{tabular}{|l|l|l|c|c|}
        \hline
        \rowcolor{blue!30}
        \textbf{\#} & \textbf{Description} & \textbf{Value} & \textbf{\S} & \textbf{Page} \\
        \hline \stepcounter{dtp}
        \thedtp & General Aperture Limitations (horizontal and vertical) & 1000 mm & \ref{ApeLim} & \pageref{ApeLim} \\
        \hline \stepcounter{dtp}
        \thedtp & Starting in the Accelerator Structure at Element Number & 1 & \ref{StrInp} & \pageref{StrInp} \\
        \hline \stepcounter{dtp}
        \thedtp & Number of Turns in the forward Direction & 1 & \ref{TraPar} & \pageref{TraPar} \\
        \cline{1-3} \stepcounter{dtp}
        \thedtp & Initial horizontal Amplitude & 0.001 mm & & \\
        \cline{1-3} \stepcounter{dtp}
        \thedtp & Horizontal and vertical Phase Space Coupling Switches on & 1 & & \\
        \cline{1-3} \stepcounter{dtp}
        \thedtp & Flat Bottom, Ramping and Flat Top Printout after Turn Number & 1 & & \\
        \cline{1-3} \stepcounter{dtp}
        \thedtp & Printout of Coordinates (file 6) after Turn Number & 10000 & & \\
        \hline \stepcounter{dtp}
        \thedtp & Kinetic Energy [MeV] of the Reference Particle & $ 10^{-6} $ & \ref{IniCoo} & \pageref{IniCoo} \\
        \hline \stepcounter{dtp}
        \thedtp & Harmonic Number & 1 & \ref{SynOsc} & \pageref{SynOsc} \\
        \cline{1-3} \stepcounter{dtp}
        \thedtp & Momentum Compaction Factor & 0.001 & & \\
        \cline{1-3} \stepcounter{dtp}
        \thedtp & Length of the Machine & 1 km & & \\
        \cline{1-3} \stepcounter{dtp}
        \thedtp & Mass of the Particle (Proton) & 938.2723128 $ \mathrm{MeV} / \mathrm{c}^2 $ & & \\
        \cline{1-3} \stepcounter{dtp}
        \thedtp & Momentum Correction Factor for Distance in Phase Space & 1 & & \\
        \cline{1-3} \stepcounter{dtp}
        \thedtp & Path-length Correction Factor for Distance in Phase Space & 1 & & \\
        \hline \stepcounter{dtp}
        \thedtp & Averaging Turn Interval for Post-processing & 1 & \ref{PosPro} & \pageref{PosPro} \\
        \hline
    \end{tabular}
    \normalsize
\end{table}

\newpage

% ================================================================================================================================ %
\section{Default Size Parameters} \label{DSP}

For large machines the arrays holding the machine parameters might have to be increased.
The size of each of the dimensions of the arrays is therefore defined as a parameter.
This can be done by compiling with the \texttt{BIGNPART}, \texttt{HUGENPART}, \texttt{BIGNBLZ}, or \texttt{HUGENBLZ} flags.
The default values are adjusted to allow the treatment of a full LHC lattice: the tracking version uses 50 Mb and the DA version 400 Mb.

\newcounter{dsp} \setcounter{dsp}{0}

\bigskip
\begin{table}[h]
    \caption{Default Size Parameters}
    \label{T-DSP}
    \scriptsize
    \centering
    \renewcommand{\arraystretch}{1.5}
    \begin{tabular}{|c|l|c|c|c|c|}
        \hline
        \rowcolor{blue!30}
        \textbf{\#} & \textbf{Description} & \textbf{Value} & \textbf{Name} & \textbf{\S} & \textbf{Page} \\
        \hline \stepcounter{dsp}
        \thedsp & Maximum Number of Coordinates used in the Correction Routines & 6 & \texttt{MPA} & & \\
        \hline \stepcounter{dsp}
        \thedsp & Number of Single Elements & 750 & \texttt{NELE} & \ref{SinEle} & \pageref{SinEle} \\
        \hline \stepcounter{dsp}
        \thedsp & Number of Blocks of Linear Elements & 160 & \texttt{NBLO} & \ref{BloDef} & \pageref{BloDef} \\
        \cline{1-4} \stepcounter{dsp}
        \thedsp & Number of Linear Elements per Block & 100 & \texttt{NELB} & & \\
        \hline \stepcounter{dsp}
        \thedsp & Total Number of Elements in the Structure & 15000 & \texttt{NBLZ} & \ref{StrInp} & \pageref{StrInp} \\
        \cline{1-4} \stepcounter{dsp}
        \thedsp & Number of Accelerator Super-periods & 16 & \texttt{NPER} & & \\
        \hline \stepcounter{dsp}
        \thedsp & Total Number of Random Values & 300000 & \texttt{NZFZ} & \ref{FluNum} & \pageref{FluNum} \\
        \cline{1-4} \stepcounter{dsp}
        \thedsp & Number of Random Values for the basic Set of Nonlinear Elements & 280000 & \texttt{NRAN} & & \\
        \hline \stepcounter{dsp}
        \thedsp & Number of Random Values for inserted Nonlinear Elements & 20000 & & \ref{OrgRan} & \pageref{OrgRan} \\
        \cline{1-4} \stepcounter{dsp}
        \thedsp & Number of Random Values for each Inserted Nonlinear Element & 500 & \texttt{MRAN} & & \\
                & Number of Nonlinear Elements that can be inserted & 20 & & & \\
        \hline \stepcounter{dsp}
        \thedsp & Limit Number of Particles for Vectorisation & 64 & \texttt{NPART} & & \\
        \hline \stepcounter{dsp}
        \thedsp & Maximum Number of Elements for Combined Tasks & 100 & \texttt{NCOM} & \ref{ComEle} & \pageref{ComEle} \\
        \hline \stepcounter{dsp}
        \thedsp & Maximum Resonance Compensation Order & 5 & \texttt{NRCO} & \ref{ComEle} & \pageref{ComEle} \\
        \hline \stepcounter{dsp}
        \thedsp & Total Number of Data for Processing & 20000 & \texttt{NPOS} & \ref{PosPro} & \pageref{PosPro} \\
        \cline{1-4} \stepcounter{dsp}
        \thedsp & Number of Intervals for Calculation of Lyapunov Exponents & 10000 & \texttt{NLYA} & & \\
        \cline{1-4} \stepcounter{dsp}
        \thedsp & Number of Intervals for Calculation of Invariants & 1000 & \texttt{NINV} & & \\
        \cline{1-4} \stepcounter{dsp}
        \thedsp & Number of Data for Plotting & 20000 & \texttt{NPLO} & & \\
        \hline \stepcounter{dsp}
        \thedsp & Maximum Pole Order of Multipole Block & 11 & \texttt{MMUL} & \ref{MulCoe} & \pageref{MulCoe} \\
        \hline \stepcounter{dsp}
        \thedsp & Maximum Number of extra Parameters of the DA Map & 10 & \texttt{MCOR} & \ref{DifAlg} & \pageref{DifAlg} \\
        \hline \stepcounter{dsp}
        \thedsp & Maximum Order of DA Calculation & 15 & \texttt{NEMA} & \ref{DifAlg} & \pageref{DifAlg} \\
        \hline \stepcounter{dsp}
        \thedsp & Maximum Number of Monitors for Micado Closed Orbit Correction & 600 & \texttt{NMON1} & \ref{OrbCorr} & \pageref{OrbCorr} \\
        \hline \stepcounter{dsp}
        \thedsp & Maximum Number of Correctors for Micado Closed Orbit Correction & 600 & \texttt{NCOR1} & \ref{OrbCorr} & \pageref{OrbCorr} \\
        \hline \stepcounter{dsp}
        \thedsp & Maximum Number of Beam--Beam Elements & 350 & \texttt{NBB} & \ref{BeamBeam} & \pageref{BeamBeam} \\
        \hline \stepcounter{dsp}
        \thedsp & Maximum Number of Slices for 6D Beam--Beam Kick & 99 & \texttt{MBEA} & \ref{BeamBeam} & \pageref{BeamBeam} \\
        \hline \stepcounter{dsp}
        \thedsp & Maximum Number of ``Phase Trombone'' Elements & 20 & \texttt{NTR} & ref{PT} & \pageref{PT} \\
        \hline
    \end{tabular}
    \normalsize
\end{table}
