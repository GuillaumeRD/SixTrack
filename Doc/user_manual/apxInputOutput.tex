\chapter{Input and Output Files} \label{Files}

The program uses a couple of files for its input and output procedures.

\begin{center}
\begin{longtable}{|c|c|c|c|c|}
  \caption{List of Input and Output Files.}\\
  \hline
  \rule[-5mm]{0mm}{12.5mm} {\bf File Unit} & {\bf Input} & {\bf Output} & {\bf File Type} & {\bf Contents} \\
  \hline
  \endfirsthead
  \hline
  \rule[-5mm]{0mm}{12.5mm} {\bf File Unit} & {\bf Input} & {\bf Output} & {\bf File Type} & {\bf Contents} \\
  \hline
  \endhead
  
%  \hline
%\rule[-1.25mm]{0mm}{7.5mm}
  \rule[-3.75mm]{0mm}{10mm}
  2 & X & & Ascii & Geometry and Strength Parameters \\
  \hline \rule[-3.75mm]{0mm}{10mm}
  3 & X & & Ascii & Tracking Parameters \\*
  \hline \rule[-3.75mm]{0mm}{10mm} 4 & & X & Ascii & Geometry and
  strength Parameters (format as file \#
  2) \\
  \hline \rule[-3.75mm]{0mm}{10mm}
  6 & & X & Ascii & Input Parameters and Analysis of Data \\
  \hline \rule[-3.75mm]{0mm}{10mm}
  8 & X & & Ascii & Name, hor., ver. Misalignment and Tilt \\
  \hline \rule[-1.25mm]{0mm}{7.5mm}
  9 & & X & Ascii & Internally used multipoles \\*
  \rule[-3.7mm]{0mm}{7.5mm}
  & & & & Format: $a16,\ 2 \times \{6 \times (1p,3d23.15),\
  (1p,2d23.15)\}$ \\ 
  \hline \rule[-3.75mm]{0mm}{10mm}
  10 & X & X & Ascii & Summary of Post--processing (auxiliary) \\
  \hline \rule[-3.75mm]{0mm}{10mm}
  11 & & X & Ascii & This file is used to dump linear \\*
  \rule[-3.7mm]{0mm}{7.5mm}
  & & & & coupling parameters at locations of choice \\
  \hline \rule[-1.25mm]{0mm}{7.5mm}
  12 & & X & Ascii & End Coordinates of both Particles \\*
  \rule[-3.7mm]{0mm}{7.5mm}
  & & & & Format: ($15 \times F10.6$) \\
  \hline \rule[-3.75mm]{0mm}{10mm}
  13 & X & & Ascii & Start Coordinates for a Prolongation \\
  \hline \rule[-1.25mm]{0mm}{7.5mm}
  14 & & X & Ascii & Horizontal FFT Spectrum for detailed \\*
  \rule[-3.7mm]{0mm}{7.5mm}
  & & & & Analysis; Format:  ($2 \times F10.6$) \\
  \hline \rule[-1.25mm]{0mm}{7.5mm}
  15 & & X & Ascii & Vertical FFT Spectrum for detailed \\*
  \rule[-3.7mm]{0mm}{7.5mm}
  & & & & Analysis; Format: ($2 \times F10.6$) \\
  \hline \rule[-1.25mm]{0mm}{7.5mm}
  16 & X & & Ascii & External multipole errors \\*
  \rule[-3.7mm]{0mm}{7.5mm}
  & & & & Format: $a16,\ 2 \times \{6 \times (1p,3d23.15),\
  (1p,2d23.15)\}$ \\ 
  \hline \rule[-1.25mm]{0mm}{7.5mm}
  17 & & X & Ascii & Additional Map at \\*
  \rule[-3.7mm]{0mm}{7.5mm}
  & & & & location of interest \\
%  \hline
%  \hline \rule[-5mm]{0mm}{12.5mm} {\bf File Unit} & {\bf Input} & {\bf Output} & {\bf File Type} & {\bf Contents}
%  \\
  \hline \rule[-1.25mm]{0mm}{7.5mm}
  18 & & X & Ascii & One--Turn Map with Differential \\*
  \rule[-3.7mm]{0mm}{7.5mm}
  & & & & Algebra \\
  \hline \rule[-3.75mm]{0mm}{10mm}
  19 & X & X & Ascii & Internal use for Differential Algebra \\
  \hline \rule[-3.75mm]{0mm}{10mm}
  20 & & X & Meta--file & PS--file of selected Plots \\
  \hline \rule[-1.25mm]{0mm}{7.5mm}
  21 & & X & Ascii & Factorisation of the one--turn \\*
  \rule[-3.7mm]{0mm}{7.5mm}
  & & & & map \\
  \hline \rule[-1.25mm]{0mm}{7.5mm}
  22 & & X & Ascii & Transformation in the \\*
  \rule[-3.7mm]{0mm}{7.5mm}
  & & & & Normal Form coordinates \\
  \hline \rule[-1.25mm]{0mm}{7.5mm}
  23 & & X & Ascii & Hamiltonian in \\*
  \rule[-3.7mm]{0mm}{7.5mm}
  & & & & action variables \\
  \hline \rule[-1.25mm]{0mm}{7.5mm}
  24 & & X & Ascii & Tune--shift in action \\*
  \rule[-3.7mm]{0mm}{7.5mm}
  & & & & coordinates \\
  \hline \rule[-1.25mm]{0mm}{7.5mm}
  25 & & X & Ascii & Tune--shift in Cartesian \\*
  \rule[-3.7mm]{0mm}{7.5mm}
  & & & & coordinates \\
  \hline \rule[-3.75mm]{0mm}{10mm}
  26 & & X & Ascii & NAGLIB log--file \\
  \hline \rule[-3.75mm]{0mm}{10mm}
  27 & & X & Ascii & Name, hor., ver. Misalignment and Tilt \\
  \hline \rule[-1.75mm]{0mm}{10mm}
  28 & & X & Ascii & Horizontal closed orbit displacement, \\*
  \rule[-3.7mm]{0mm}{7.5mm}
  & &   &       & measured at monitors \\
  \hline \rule[-1.75mm]{0mm}{10mm}
  29 & & X & Ascii & Vertical closed orbit displacement, \\*
  \rule[-3.7mm]{0mm}{7.5mm}
  & &   &       & measured at monitors \\
%  \hline
%  \hline \rule[-5mm]{0mm}{12.5mm} {\bf File Unit} & {\bf Input} & {\bf Output} & {\bf File Type} & {\bf Contents} \\
  \hline \rule[-3.75mm]{0mm}{10mm}
  30 & X & & Ascii & Name, Random strength, misalignments and tilt \\
  \hline \rule[-3.75mm]{0mm}{10mm}
  31 & & X & Ascii & Name, Random strength, misalignments and tilt \\
  \hline \rule[-3.75mm]{0mm}{10mm}
  32 & X & X & Binary & Binary dump of full accelerator description \\
  \hline \rule[-3.75mm]{0mm}{10mm}
  33 & X & & Ascii & Guess values for 6D closed orbit search \\
  \hline \rule[-3.75mm]{0mm}{10mm}
  34 & & X & Ascii & Multipole strength and linear lattice
    parameters~\cite{SODD} \\ 
  \hline \rule[-1.25mm]{0mm}{7.5mm}
  90 -- k & & X & Binary & Tracking Data (not singletrackfile)\\*
  \rule[-3.7mm]{0mm}{7.5mm}
  & & & & $ 0 <= k <= 31 $\\
  \hline \rule[-1.25mm]{0mm}{7.5mm}
  90 & & X & Binary & Tracking Data (singletrackfile)\\*
  \rule[-3.7mm]{0mm}{7.5mm}
  & & & & \texttt{singletrackfile.dat}\\
  \hline \rule[-3.75mm]{0mm}{10mm}
  92 & & X & Ascii & Checkpoint/Restart only: \\*
  \rule[-3.7mm]{0mm}{7.5mm}
  & & & & Program ``standard output'' (lout) \\
  \hline \rule[-3.75mm]{0mm}{10mm}
  93 & & X & Ascii & Checkpoint/Restart only: Log file \\
  \hline \rule[-3.75mm]{0mm}{10mm}
  94 & & X & Ascii & Checkpoint/Restart only: Temp file for \\* 
  \rule[-3.7mm]{0mm}{7.5mm}
  & & & &resetting binary tracking data file(s) \\
  \hline \rule[-3.75mm]{0mm}{10mm}
  95 & X & X & Ascii & Checkpoint/Restart only: Data file 1 \\
  \hline \rule[-3.75mm]{0mm}{10mm}
  96 & X & X & Ascii & Checkpoint/Restart only: Data file 2 \\
  \hline \rule[-3.75mm]{0mm}{10mm}
  98 & & X & Ascii & 6D coordinates at Cavity (1p,6(2x,e25.18)) \\
  \hline \rule[-3.75mm]{0mm}{10mm}
  664 & X &  & Ascii & DYNK reading FUN FILE(LIN) \\*
  \rule[-3.7mm]{0mm}{7.5mm}
  & & & & (only during initialization) \\
  \hline \rule[-3.75mm]{0mm}{10mm}
  665 & & X & Ascii & DYNK output file \texttt{dynksets.dat} \\
  \hline \rule[-3.75mm]{0mm}{10mm}
  2001001 & & X & Ascii & FMA output file \texttt{fma\_sixtrack} \\\hline \rule[-3.75mm]{0mm}{10mm}
  200101+i*10 & & X & Ascii & FMA output file \texttt{NORM\_*}, \\*
  \rule[-3.7mm]{0mm}{7.5mm}
  & & & & where $i=1,\ldots,\mathrm{number \ of \ FMAs}$ \\\hline
\end{longtable}
\end{center}

%\clearpage

In addition to those files listed in the table, DUMP uses arbitary file unit numbers as determined by the input file. The collimation module also uses many input/output files at various units, which are not listed here.
